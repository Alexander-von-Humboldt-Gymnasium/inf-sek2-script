\clearpage

\lehead[]{\sf\hspace*{-2.00cm}\textcolor{white}{\colorbox{lightblue}{\parbox[c][0.70cm][b]{1.60cm}{
\makebox[1.60cm][r]{\thechapter}\\ \makebox[1.60cm][r]{ÜBUNG}}}}\hspace{0.17cm}\textcolor{lightblue}{\chaptertitle}}
\rohead[]{\textcolor{lightblue}{\chaptertitle}\sf\hspace*{0.17cm}\textcolor{white}{\colorbox{lightblue}{\parbox[c][0.70cm][b]{1.60cm}{\thechapter\\
ÜBUNG}}}\hspace{-2.00cm}}
%\chead[]{}
\rehead[]{\textcolor{lightblue}{AvHG, Inf, My}}
\lohead[]{\textcolor{lightblue}{AvHG, Inf, My}}

\section{Dialoge -- Übungen}

Bearbeite die Aufgaben in der Datei \myFile{Dialoge.java}, welche du im
Kurs-Repository findest.

\subsection{Aufgabe 1: Zahl zwischen 1 und 100}

\begin{compactenum}[a)]
\item Schreibe eine Methode, die überprüft ob eine als Parameter übergebene
ganze Zahl zwischen 1 und 100 liegt. Das Ergebnis wird als boolescher Wert
zurück gegeben.

\item Der Benutzer wird in einem Eingabe-Dialog nach einer Zahl gefragt, die an
die Methode aus Teil (a) übergeben wird. Das Ergebnis wird in einem
Ausgabe-Dialog angezeigt.
\end{compactenum}


\subsection{Aufgabe 2: Zahl kleiner als 10 oder größer als 20}

\begin{compactenum}[a)]
\item Schreibe eine Methode, die überprüft ob eine als Parameter übergebene
 Fließkommazahl kleiner als 10 oder größer als 20 ist. Das Ergebnis wird als
 boolescher Wert zurück gegeben.

\item Der Benutzer wird in einem Eingabe-Dialog nach einer Zahl gefragt, die an
die Methode aus Teil (a) übergeben wird. Das Ergebnis wird in einem
Ausgabe-Dialog angezeigt.
\end{compactenum}


\subsection{Aufgabe 3: Gerade Zahl}

\begin{compactenum}[a)]
\item Schreibe eine Methode, die überprüft ob eine als Parameter übergebene
ganze Zahl gerade (das heißt durch 2 teilbar) ist. Das Ergebnis wird als
boolescher Wert zurück gegeben.

\item Der Benutzer wird in einem Eingabe-Dialog nach einer Zahl gefragt, die an
die Methode aus Teil (a) übergeben wird. Das Ergebnis wird in einem
Ausgabe-Dialog angezeigt.
\end{compactenum}


\subsection{Aufgabe 4: Umwandlung von Notenpunkten in Noten}

\begin{compactenum}[a)]

\item Schreibe eine Methode, die eine Zensur in Punktangabe in die übliche
 Notenbezeichnung aus der Mittelstufe umrechnet:

\begin{tabular}{rcl}
0 & $\rightarrow$ & 6 \\
1 bis 3 & $\rightarrow$ & 5\\
4 bis 6 & $\rightarrow$ & 4\\
7 bis 9 & $\rightarrow$ & 3\\
10 bis 12 & $\rightarrow$ & 2\\
13 bis 15 & $\rightarrow$ & 1\\
Alle anderen Werte & $\rightarrow$ & -1 (als Fehlerwert)\\
\end{tabular}

\item Der Benutzer wird in einem Eingabe-Dialog nach einer Zahl gefragt, die an
die Methode aus Teil (a) übergeben wird. Das Ergebnis wird in einem Ausgabe-Dialog angezeigt.

\item Der Benutzer soll nacheinander mehrere Zahlen überprüfen können ohne
erneut auf den Button zu drücken. Rufe den Code aus Teilaufgabe (b) in einer
Schleife wiederholt auf, bis der Benutzer den \myPMI{Abbrechen}-Button drückt.
\end{compactenum}


\subsection{Aufgabe 5: Mathe-Trainer}

\begin{compactenum}[a)]

\item In einem Eingabe-Dialog werden nacheinander die
folgenden Rechenaufgaben angezeigt, die der
Benutzer lösen muss:

\vspace{3mm}

\begin{tabular}{ccc}
5*7 & 6*7 & 7*7\\
5*8 & 6*8 & 7*8\\
5*9 & 6*9 & 7*9\\
\end{tabular}

Verwende dazu zwei ineinander geschachtelte for- Schleifen. Die eine Schleife
zählt die erste Zahl von 5 bis 7 hoch. Die andere Schleife zählt die zweite Zahl
von 7 bis 9 hoch. Nachdem der Benutzer seine Lösung eingegeben hat, wird die
Korrektheit der Lösung überprüft. In einem Ausgabedialog wird dann entweder
\glqq Korrekt.\grqq\ ausgegeben oder es wird ausgegeben \glqq Falsch. Die
richtige Lösung ist \emph{Lösung}.\grqq

\item Erweitere das Programm so, dass der Benutzer nach Eingabe einer falschen
 Lösung so lange wiederholt dieselbe Aufgabe erhält, bis er die Lösung korrekt
 angegeben hat.
\end{compactenum}


\subsection{Aufgabe 6: Lösung quadratischer Gleichungen}

Gleichungen der Form $x^2 + p \cdot x + q = 0$ können mit der $p$-$q$-Formel
gelöst werden.

Die p-q-Formel liefert zwei Lösungen:

\begin{minipage}{0.32\textwidth}
$$
x_1 = - \frac{p}{2} + \sqrt{\left(\frac{p}{2}\right)^2 - q}
$$
\end{minipage}
und 
\begin{minipage}{0.32\textwidth}
$$
x_2 = - \frac{p}{2} - \sqrt{\left(\frac{p}{2}\right)^2 - q}
$$
\end{minipage}

Frage den Benutzer in zwei Eingabe-Dialogen nach einem Wert für $p$ und $q$.
Wandle die Benutzereingaben in Fließkommazahlen um und berechne zunächst den
Wert unter der Wurzel. Falls dieser Wert negativ ist, wird in einem
Ausgabe-Dialog \glqq keine Lösung\grqq\ ausgegeben, da man keinen Wert aus
einer negativen Zahl ziehen kann. Falls der Wert 0 ist, gibt es nur eine Lösung,
die in einem Ausgabe-Dialog angezeigt wird, nämlich $x = - \frac{p}{2}$. Falls
der Wert positiv ist, gibt es zwei Lösungen. Berechne beide Lösungen und gib
sie in einer zusammenhängenden Zeichenkette mit einem einzigen Ausgabe-Dialog
aus.

Teste dein Programm mit den Testdaten in der folgenden Tabelle.

\begin{minipage}{0.58\textwidth}
\begin{tabular}{|l|l|}\hline
\textbf{Aufgabe} & \textbf{Lösung}
\\ \hline
$x^2 - 4 \cdot x - 5 = 0$ & $x_1 = 5$, $x_2 = -1$
\\ 
$x^2 + 7,9 \cdot x + 2,5 = 0$ & $x_1 = 0,330\ldots$, $x_2 = -7,569\ldots$
\\ 
$x^2 + 4,6 \cdot x - 1 = 0$ & $x_1 = 0,207\ldots$, $x_2 = -4,807\ldots$
\\ 
$x^2 + 4 \cdot x + 4 = 0$ & $x = -2$
\\ 
$x^2 - 5 \cdot x + 17 = 0$ & keine Lösung
\\ 
$x^2 + 0 \cdot x + 6 = 0$ & keine Lösung
\\ \hline
\end{tabular}
\end{minipage}\hfill
\begin{minipage}{0.42\textwidth}
\textbf{Mathematische Funktionen}

\begin{lstlisting}
double wurzel = Math.sqrt(3); æ// `$\sqrt{3}$`
ædouble hoch = Math.pow(4,2);  æ// `$4^2$`
\end{lstlisting}
\end{minipage}


\subsection{Aufgabe 7: Body Mass Index (BMI)}

Der Body Mass Index wird von
Ernährungswissenschaftlern verwendet, um festzustellen
ob eine Person Über- oder Untergewicht hat. Der Body
Mass Index berechnet sich nach folgender Formel:

$$
\textrm{BMI} = \frac{\textrm{Gewicht in kg}}{(\textrm{Größe in m})^2}
$$

Erwachsene gelten als normalgewichtig, wenn ihr BMI zwischen 18,5 und 25 liegt.
Wenn ihr BMI über 25 liegt, haben sie Übergewicht. Wenn er unter 18,5 liegt,
haben sie Untergewicht. Frage den Benutzern in zwei Eingabedialogen zuerst nach
seinem Gewicht und dann nach seiner Größe (Achtung: Die Größe muss in Metern
eingeben werden, also als Kommawert!). Wandle beide Werte in
Fließkommazahlen um, berechne den BMI und gib in einem Ausgabedialog aus, ob der Benutzer
Übergewicht, Untergewicht oder Normalgewicht hat.


\subsection{Aufgabe 8: Niedersachsenticket}

Ein Niedersachsenticket ist einen Tag gültig und kostet 28€ . Maximal 5 Personen
können auf dem Ticket an Wochentagen in Regionalzügen durch Bremen, Hamburg und
Niedersachsen fahren. Wie viel eine Person z.B. für eine Tagesfahrt von Bremen
nach Hamburg zahlt, hängt von der Anzahl der Mitfahrer ab. Wenn beispielsweise 2
Personen nach Hamburg fahren, zahlt jeder 28€ : 2 = 14€. Wenn statt dessen 6
Personen mitfahren, müssen zwei Ticket gekauft werden. Jeder zahlt dann 56€ : 6
= 9,33€.

\begin{compactenum}[a)]
\item Schreibe eine Methode, die als Parameter die Anzahl der Personen erhält
und als Rückgabewert den Fahrpreis für jede einzelne Person ermittelt. Welchen
Datentyp sollte man für den Parameter und den Rückgabewert wählen? 

\item Der Benutzer wird in einem Eingabe-Dialog nach der Anzahl der Personen
gefragt, die an die Methode aus Teil a) übergeben wird. Der berechnete
Fahrpreis wird in einem Ausgabe-Dialog angezeigt.

\item Der Benutzer soll nacheinander mehrere Zahlen eingeben können ohne erneut
auf den Button zu drücken. Rufe den Code aus Teilaufgabe b) in einer Schleife
wiederholt auf, bis der Benutzer den Abbrechen-Button drückt.
\end{compactenum}