\clearpage

\lehead[]{\sf\hspace*{-2.00cm}\textcolor{white}{\colorbox{lightblue}{\parbox[c][0.70cm][b]{1.60cm}{
\makebox[1.60cm][r]{\thechapter}\\ \makebox[1.60cm][r]{ÜBUNG}}}}\hspace{0.17cm}\textcolor{lightblue}{\chaptertitle}}
\rohead[]{\textcolor{lightblue}{\chaptertitle}\sf\hspace*{0.17cm}\textcolor{white}{\colorbox{lightblue}{\parbox[c][0.70cm][b]{1.60cm}{\thechapter\\
ÜBUNG}}}\hspace{-2.00cm}}
%\chead[]{}
\rehead[]{\textcolor{lightblue}{AvHG, Inf, My}}
\lohead[]{\textcolor{lightblue}{AvHG, Inf, My}}

\section{Kontrollfragen zum Text}

\subsection*{Fragen zum Abschnitt \glqq Was vorher geschah\grqq}
\begin{compactenum}[a)]
\item Erkläre den Unterschied zwischen eigenen Produkten einer Firma im
Gegensatz zu einer Auftragsarbeit.
\item Überlege welche Vor- und Nachteile die Entwicklung eigener Produkte für
Firma besitzt.
\item Beschreibe mögliche Einsatzgebiete für das beschriebene
Video-Überwachungssystem.
\end{compactenum}

\subsection*{Fragen zum Abschnitt \glqq Der Vertrag wird abgeschlossen\grqq}
\begin{compactenum}[a)]
\item Welche Positionen haben die Mitarbeiter, die an den Vertragsverhandlungen
beteiligt sind, in der Firma LogoSoft? Welche Aufgaben kommen ihnen während der
Verhandlung wahrscheinlich zu? 
\item Welche Maßnahmen werden getroffen, um die Kosten des Projekts möglichst
gering zu halten? Welche Risiken bringen diese Maßnahmen mit sich? Würdest du
als Projektleiter oder Geschäftsleiter diese Risiken in Kauf nehmen?
\item Das noch nicht entwickelte Software-System wird an die Firma BlueEye zu
einem festen Preis verkauft. Welches Risiko birgt ein Festpreis für die Firma
LogoSoft? Was könnten die Gründe dafür sein, das sich die Geschäftsleitung
trotz des Risikos zu einem Festpreis bereit erklärt?
\end{compactenum}

\subsection*{Fragen zum Abschnitt \glqq Das Projekt beginnt\grqq}
\begin{compactenum}[a)]
\item Liste die Stärken und Schwächen des Projektteams „Video-Überwachung“ auf.
Erfüllt das Team die Bedingungen, von denen der Senior Developer bei seiner
Grob-Planung ausgegangen ist (siehe Abschnitt \glqq Der Vertrag wird
abgeschlossen\grqq)?
\item Wer trägt die Verantwortung für die schlechte Zusammensetzung des Teams?
\item Beschreibe, welche Teilaufgaben in der Planungsphase eines Projekts
erledigt werden müssen. Wer führt diese Aufgaben aus?
\end{compactenum}

\subsection*{Fragen zum Abschnitt \glqq Ablauf des ersten halben Jahres\grqq}
\begin{compactenum}[a)]
\item Welche Vorbereitungen sind nötig, ehe ein Softwareentwickler mit der
Programmierung beginnen kann? Wie viel Zeit benötigen die Entwickler im Team
\glqq Video-Überwachung\grqq\ für die Vorbereitung?
\item Aus welchen Gründen verzögert sich der Ablauf des Projektes? Fallen dir
außer den beiden im Abschnitt genannten Gründen noch weitere Gründe ein, die
vermutlich zu der Verzögerung beigetragen haben?
\item Das Projekt wird einen Monat vor dem geplanten Projektende mit zwei
zusätzlichen Softwareentwicklern verstärkt. Was sagt dieser späte Zeitpunkt
über die Projektorganisation aus? Welche Maßnahmen hätten helfen können, die
Unhaltbarkeit des Liefertermins früher zu entdecken?
\end{compactenum}

\subsection*{Fragen zum Abschnitt \glqq Das Projekt muss verlängert werden\grqq}
\begin{compactenum}[a)]
\item Dem Neuzugang von der Universität wurde sofort eine sehr anspruchsvolle
Aufgabe übertragen, der er nicht gewachsen war. Was sagt dies über die
sogenannte \glqq Praxisnähe\grqq\ einer Universitätsausbildung aus? Glaubst du,
dass die Aufgabe zumutbar war? Hätte ein anderer, durchschnittlicher
Universitätsabsolvent die Aufgabe vermutlich besser gelöst?
\item Eva stellt fest, dass die Zeitschätzung für ihre Komponente um ein Drittel
zu niedrig war. Eine solche Fehleinschätzung ist bei kleinen Unternehmen an der
Tagesordnung. Welche finanziellen Konsequenzen kann eine solche
Fehleinschätzung für die Firmen haben?

Kannst du dir vorstellen, welche psychologischen Gründe es dafür gibt, dass die
Projektplaner die benötigte Zeit immer wieder zu niedrig einschätzen? 
\item Im Text steht: \glqq Fast die gesamte Softwareentwicklungsabteilung ist
jetzt in das Projekt eingebunden.\grqq\ Zähle die Anzahl der Mitarbeiter, die
jetzt am Projekt beteiligt sind, zusammen (siehe auch Abschnitte \glqq Das
Projekt beginnt\grqq\ und \glqq Ablauf des ersten halben Jahres\grqq ) und
schätze daraus die Größe der Softwareentwicklungsabteilung ab.
\end{compactenum}

\subsection*{Fragen zum Abschnitt \glqq Der Geldhahn wird zugedreht\grqq}

In der Abteilung \glqq Technik\grqq\ werden sämtliche Mitarbeiter entlassen,
die bisher für \glqq die Wartung der Geräte, die Kundenbetreuung und den Test
von fertigen Software-Komponenten\grqq\ zuständig waren.

\begin{compactenum}[a)]
\item Was hat dies für organisatorische Konsequenzen für die Firma? In der
Abteilung Technik sind nur die Softwareentwickler übrig geblieben. Könnten
Mitarbeiter aus anderen Abteilungen (Vertrieb, Buchhaltung, usw.) die Aufgaben
der Techniker übernehmen? 
\item Wenn die Aufgaben der entlassenen Mitarbeiter in Zukunft von den
Softwareentwicklern übernommen werden, ist dann zu erwarten, dass die Zeit für
die Bearbeitung einer Aufgabe länger oder kürzer dauert als bisher? Bedenke,
dass die Softwareentwickler zwar eine Universitätsausbildung besitzen aber
bisher mit den Aufgaben der \glqq einfachen\grqq\ Techniker nicht belastet
wurden.
\item Welche finanziellen Konsequenzen hat es für die Firma, wenn die Aufgaben
der entlassenen Mitarbeiter von den Softwareentwicklern mit übernommen werden
müssen? Berücksichtige, dass die Softwareentwickler eine bessere Ausbildung als
die entlassenen Mitarbeiter besitzen.
\end{compactenum}

\subsection*{Fragen zum Abschnitt \glqq Technische Mängel\grqq}
\begin{compactenum}[a)]
\item Das Video-Überwachungssystem hat technische Mängel, die man nur durch die
Erstellung eines Prototyps hätte verhindern können. Erkläre, was in diesem
Zusammenhang ein \glqq Prototyp\grqq\ ist.
\item Zu welchem Zeitpunkt hätte der Prototyp erstellt werden müssen? 
\item Die Firma LogoSoft zahlt die Gehälter an die Mitarbeiter verspätet aus.
Welche Nachteile können einem Angestellten daraus erwachsen, wenn das Gehalt
nicht zum erwarteten Termin auf dem Girokonto eingeht? Welche Konsequenzen
würdest du als Mitarbeiter der Firma LogoSoft daraus ziehen?
\end{compactenum}

\subsection*{Fragen zum Abschnitt \glqq Die Test-Phase\grqq}
\begin{compactenum}[a)]
\item Im Text ist davon die Rede, das \glqq Langzeit-Tests\grqq\ durchgeführt
werden. Das Video-Überwachungssystem wird also nicht nur für eine halbe Stunde
gestartet und dann wieder abgeschaltet, sondern die Software muss einen
längeren Zeitraum hindurch ohne Fehler laufen. Was schätzt du, über wie lange
Zeit die Software auf einem einzelnen Rechner Bilder aufzeichnen muss, bis man
sagen kann, das das System fehlerfrei funktioniert: mehrere Stunden, mehrere
Tage oder mehrere Wochen?
\item Erkläre den zyklischen Ablauf innerhalb der Testphase mit deinen eigenen
Worten.
\item Wie lange dauert die Testphase für das Projekt \glqq
Video-Überwachung\grqq ? Wie lange dauerte die vorhergehende Programmierphase?
\end{compactenum}

\subsection*{Abschließende Auswertung}
\begin{compactenum}[a)]
\item Welche verschiedenen Arbeitsphasen gibt es bei der Entwicklung eines
großen Softwaresystems? Wie lange dauerten die einzelnen Phasen in dem Beispiel
\glqq Video-Überwachung\grqq\ ungefähr?
\item Bei der Entwicklung des Video-Überwachungssystems ging einiges schief. In
welcher Arbeitsphase wurden die größten Fehler gemacht? 
\item Wie groß ist der Zeitunterschied zwischen der geplanten und der
tatsächlichen Dauer des Projektes? 
\item Wie groß ist der Unterschied zwischen dem geplanten und dem tatsächlichen
Arbeitsaufwand für das Projekt? Schätze dazu ab, wie viele Monate ein einziger
Softwareentwickler für die gleiche Arbeit gebraucht hätte (nach Plan und
tatsächlich).
\end{compactenum}

