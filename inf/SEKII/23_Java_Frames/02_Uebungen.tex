\clearpage

\lehead[]{\sf\hspace*{-2.00cm}\textcolor{white}{\colorbox{lightblue}{\parbox[c][0.70cm][b]{1.60cm}{
\makebox[1.60cm][r]{\thechapter}\\ \makebox[1.60cm][r]{ÜBUNG}}}}\hspace{0.17cm}\textcolor{lightblue}{\chaptertitle}}
\rohead[]{\textcolor{lightblue}{\chaptertitle}\sf\hspace*{0.17cm}\textcolor{white}{\colorbox{lightblue}{\parbox[c][0.70cm][b]{1.60cm}{\thechapter\\
ÜBUNG}}}\hspace{-2.00cm}}
%\chead[]{}
\rehead[]{\textcolor{lightblue}{AvHG, Inf, My}}
\lohead[]{\textcolor{lightblue}{AvHG, Inf, My}}

\section{Fenster -- Übungen}

\subsection{Aufgabe 1: \lstinline|windowClosing()|}

\begin{compactenum}[a)]
\item Erzeuge in Eclipse über Rechtsclick auf das aktuelle Package im Package
Explorer und \myPMI{New} $\rightarrow$ \myPMI{Class} eine Anwendung mit einem
eigenen Programmfenster, in dem du dein Programm von der Klasse \myClass{JFrame}
ableitest. Die Hilfsklasse \myClass{HJFrame} darf von heute an nicht mehr
verwendet werden!!!

Fange das Fenster-Ereignis \lstinline|windowClosing| ab, damit die Anwendung
vom Benutzer geschlossen werden kann (Damit Swing nicht mit seinem
Standard-Verhalten dazwischen-funkt musst du im Konstruktor
\lstinline|setDefaultCloseOperation(JFrame.DO_NOTHING_ON_CLOSE)| benutzen).

\item Erweitere deinen Eventhandler für das Ereignis \lstinline|windowClosing|
so, dass der Benutzer zuerst mit einem \lstinline|showConfirmDialog()| der
Klasse \myClass{JOptionPane} (siehe unten) gefragt wird, ob er die Anwendung
wirklich schließen möchte. Wenn er bejaht, wird die Anwendung geschlossen,
andernfalls nicht.

\item Fange die Ereignisse \lstinline|windowActivated| und
\lstinline|windowDeactivated| ab. Wenn das Anwendungsfenster aktiviert ist (das
erkennt man an der blau markierten Titelleiste), soll die Hintergrundfarbe des
Fensters gelb sein. Wenn das Fenster deaktiviert ist (graue Titelleiste), ist
die Hintergrundfarbe blau.

\item Wenn das Fenster aktiviert ist, wird in der Titelzeile der Text
\glqq Schön, dass du da bist!\grqq\ ausgegeben. Wenn das Fenster deaktiviert ist
wird der Text \glqq Warum hast du mich verlassen?\grqq\ ausgegeben.
\end{compactenum}

\subsubsection{Bestätigungs-Dialoge}

Die Klasse \myClass{JOptionPane} aus der Swing-Bibliothek besitzt die Methode
\lstinline|showConfirmDialog()| zur Abfrage einer Benutzerbestätigung:

\begin{lstlisting}
public static int showConfirmDialog(Component parentComponent, Object message)
\end{lstlisting}

Als Parameter werden die Vater-Komponente (das ist das Anwendungsfenster) und
ein Text angegeben.

Der Rückgabewert zeigt an, welchen Button der Benutzer gedrückt hat. Es sind
folgende Rückgabewerte definiert:

\bgroup
\def\arraystretch{1.2}
\begin{tabular}{ll}
\lstinline|JOptionPane.YES_OPTION| \hspace{15mm} &
(Ja-Button gedrückt)
\\
\lstinline|JOptionPane.NO_OPTION| \hspace{15mm} &
(Nein-Button gedrückt)
\\
\lstinline|JOptionPane.CANCEL_OPTION| \hspace{15mm} &
(Abbrechen-Button gedrückt)
\\
\end{tabular}
\egroup

Swing ist nicht Thread-Safe. Das bedeutet, dass der Programmierer selbst dafür
sorge tragen muss, dass nicht mehrere Threads gleichzeitig versuchen auf ein
Objekt zuzugreifen. Es ist im Zweifelsfall sinnvoll, Swing-Methoden nicht
direkt auszuführen, sondern diese an den sogenannten \emph{Event Dispatch
Thread} (EDT) zu übergeben.

Im Anwendungsfenster könnte zum Beispiel folgender Code stehen:

\begin{lstlisting}
EventQueue.invokeLater(new Runnable() {
    public void run() {
        int erg = JOptionPane.showConfirmDialog(null, 
                           "Anwendung wirklich beenden?");
        if (erg == JOptionPane.YES_OPTION) {
            æ// ...
æ        }
    }
});
\end{lstlisting}

Hinweis: Wenn du \lstinline|showConfirmDialog()| in einer anonymen inneren
Klasse (wie oben) benutzt, dann musst du als erstes Argument \lstinline|null|
anstelle von \lstinline|this| benutzen!


\subsection{Aufgabe 2: Beleidigtes Fenster}

\begin{compactenum}[a)]
\item Programmiere ein \myClass{JFrame} mit einem eigenen Eventhandler für
Fenster-Ereignisse. Die Fenster-Ereignisse können entweder in der
\myClass{JFrame}-Klasse abgefangen werden (indem das entsprechende Interface
implementiert wird) oder in einer eigenen Klasse, die sich von der Klasse
\myClass{WindowAdapter} ableitet.

Das Fenster soll zu Beginn eine Breite und eine Höhe von 500 Pixel haben. Es
hat einen gelben Hintergrund und eine schwarze Schriftfarbe. Im Fenster wird
mit Hilfe der \lstinline|paintComponent()|-Methode der Text \glqq Wehe du gehst
weg.\grqq\ angezeigt.

Im ersten Schritt wird der Eventhandler so programmiert, dass das Fenster bei
einem Klick auf den „Schließen“-Knopf geschlossen wird.

\item Fange die Fenster-Ereignisse \lstinline|windowActivated| und
\lstinline|windowDeactivated| ab.

Wenn das Fenster  deaktiviert wird, nimmt es eine rote Hintergrundfarbe an (es
wird sozusagen rot vor Ärger). Wenn es aktiviert wird, nimmt es wieder eine
gelbe Hintergrundfarbe an.

\item Wenn das Fenster deaktiviert ist, wird der Text im Fenster durch den neuen
Text \glqq Verräter, wieso gehst du einfach weg?\grqq\ ersetzt. Wenn das
Fenster aktiviert ist wird wieder der alte Text \glqq Wehe du gehst weg.\grqq\
angezeigt.

\item Wenn das Fenster deaktiviert wird, schrumpft es vor Ärger zusammen.
Reduziere die aktuelle Fensterbreite und Fensterhöhe bei jeder Deaktivierung
jeweils um 100 Pixel. Wenn die Breite des Fensters dadurch kleiner als 100
Pixel wird, beendet sich die Anwendung selbständig.

Hinweis: Es darf vereinfacht davon ausgegangen werden, dass der Benutzer die
Fenstergröße nicht manuell verstellt.
\end{compactenum}


\subsection{Aufgabe 3: \myClass{JFrame}-Template von Eclipse benutzen}

Eclipse erstellt das Grundgerüst für eine Anwendung, die von einem
\myClass{JFrame} abgeleitet wird, wenn man sich statt \myPMI{New} $\rightarrow$
\myPMI{Class} für \myPMI{New} $\rightarrow$ \myPMI{Other \ldots}
$\rightarrow$ \myPMI{Window Builder} $\rightarrow$ \myPMI{Swing Designer}
$\rightarrow$ \myPMI{JFrame} entscheidet.

Erzeuge auf diese Art und Weise eine neue Anwendung und versuche den
Programmcode zu verstehen.
