\chapter{Nebenläufige Programmierung: Threads}
\renewcommand{\chaptertitle}{Nebenläufige Programmierung: Threads}

\lehead[]{\sf\hspace*{-2.00cm}\textcolor{white}{\colorbox{lightblue}{\makebox[1.60cm][r]{\thechapter}}}\hspace{0.17cm}\textcolor{lightblue}{\chaptertitle}}
\rohead[]{\textcolor{lightblue}{\chaptertitle}\sf\hspace*{0.17cm}\textcolor{white}{\colorbox{lightblue}{\makebox[1.60cm][l]{\thechapter}}}\hspace{-2.00cm}}
%\chead[]{}
\rehead[]{\textcolor{lightblue}{AvHG, Inf, My}}
\lohead[]{\textcolor{lightblue}{AvHG, Inf, My}}

\lstset{style=myJava}

\section{Wofür benötigt man Threads?}

Threads ermöglichen es ein Programm in mehrere Teile zu unterteilen, die
unabhängig voneinander parallel ausgeführt werden. Jeder Thread (engl. für
Faden) arbeitet eigenständig in seinem eigenen Tempo. Über Variablen können die
Threads Informationen untereinander austauschen. Wenn ein Programm
beispielsweise eine lange Berechnung durchführen muss, die mehrere Minuten
dauert, so wäre es in einem \glqq normalen\grqq\ Programm während dieser Zeit
nicht möglich auf Benutzereingaben zu reagieren. Wenn man die Berechnung dagegen
in einem zweiten Thread durchführt, kann sich der Haupt-Thread weiterhin um
Benutzereingaben kümmern. Weitere Beispiele für den Einsatz von Threads:

\begin{compactitem}
\item Wenn ein Text gedruckt wird, ist dies eine Aktion die relativ lange
dauert. Wird der Druckvorgang in einem eigenen Thread durchgeführt, so steht der
Haupt-Thread dem Benutzer während des Druckvorgangs für weitere Arbeitsaufträge
zur Verfügung.

\item Während auf Eingaben des Benutzers gewartet wird, sollen im Hintergrund
aufwändige Berechnungen fortgeführt werden.
\end{compactitem}

Jede von Thread abgeleiteten Klasse muss die Methode \textbf{\lstinline|run()|}
überschreiben. Diese wird automatisch ausgeführt sobald der Thread über die
geerbte (und bei Bedarf ebenfalls überschriebene) Methode
\textbf{\lstinline|start()|} aus der Basisklasse \myClass{Thread} gestartet
wird.

Der Methode \textbf{\lstinline|start()|} kommt dabei eine besondere Aufgabe zu:
Sie sorgt dafür, dass der neue Thread aus Sicht des Betriebssystems als
eigenständiger Prozess läuft. Nur so laufen die verschiedenen Threads
tatsächlich parallel nebeneinander her. Würde man \lstinline|run()| hingegen
direkt aufrufen (was technisch gesehen durchaus möglich ist), würde diese
Methode -- wie alle anderen Methoden auch -- ganz normal ausgeführt. Und der
nächste Programmschritt würde erst dann bearbeitet, wenn \lstinline|run()| seine
Arbeit beendet hat.

Obwohl es in der Basisklasse \myClass{Thread} eine Methode
\textbf{\lstinline|stop()|} gibt, sollte man diese nicht verwenden um einen
Thread zu beenden! Vielmehr sollte man über eine boolesche Variable, die den
Zustand (läuft oder läuft nicht) des Threads festhält steuern, ob die Methode
\lstinline|run()| weiter ausgeführt werden soll. Der Thread als Ganzes wird
beendet, sobald die Methode \lstinline|run()| beendet ist. Siehe dazu das letzte
Beispiel in diesem Kapitel.

\begin{center}
\begin{minipage}{1.0\textwidth}
\begin{framed}
\textbf{Merksatz:} \lstinline|run()| muss, \lstinline|start()| kann
überschrieben werden. Und \lstinline|stop()| sollte man nicht verwenden.
\end{framed}
\end{minipage}
\end{center}


\section{Beispiel}

\begin{compactenum}[a)]
\item \textbf{Thread-Klasse}

\begin{lstlisting}
æimport javax.swing.JLabel;

public class ZaehlThread æextends Threadæ {
    JLabel lbl;
    int index;
    int zaehler = 0;

    public ZaehlThread(JLabel lbl, int index) {
        this.lbl = lbl;
        this.index = index;
    }

    @Override
    public void ærun()æ {
        while (true) {
            lbl.setText("Zaehler" + index + ": " + zaehler);
            try {
                Thread.æsleepæ(index * 1000);
            }
            catch (Exception e) {
            }
            zaehler++;
        }
    }
}
\end{lstlisting}

Erläuterungen:

\begin{compactenum}[1.]
\item Um einen Thread zu programmieren muss man eine Klasse von der Superklasse
\myClass{Thread} ableiten.
\item Die Methode \lstinline|run()| wird vom System automatisch aufgerufen
sobald der Thread gestartet wurde. Der Thread existiert bis die Methode
\lstinline|run()| beendet ist.
\item \lstinline|sleep()| ist eine statische Methode der Klasse
\myClass{Thread}, die den aktuellen Thread für die angegebene Anzahl von
Millisekunden schlafen legt (blockiert).
\end{compactenum}

\item \textbf{Frame}

\begin{lstlisting}
æimport java.awt.*;
import javax.swing.*;
import javax.swing.border.EmptyBorder;

public class ThreadBeispiel extends JFrame {
    // globale Variablen
    private static final int WIDTH = 200;
    private static final int HEIGHT = 120;
    private JLabel label[];
    private ZaehlThread thread[];

    public ThreadBeispiel(final String title) {
        super(title);
        setDefaultCloseOperation(JFrame.EXIT_ON_CLOSE);
        JPanel contentPane = new JPanel();
        contentPane.setBorder(new EmptyBorder(5, 5, 5, 5));
        contentPane.setLayout(new FlowLayout());
        contentPane.setPreferredSize(new Dimension(WIDTH, HEIGHT));
        setContentPane(contentPane);
        label = new JLabel[3];
        thread = new ZaehlThread[3];
        for (int i = 0; i < 3; i++) {
            label[i] = new JLabel();
            label[i].setFont(new Font("Arial", Font.BOLD, 20));
            contentPane.add(label[i]);
            thread[i] = ænew ZaehlThreadæ(label[i], i + 1);
            thread[i].æstart()æ;
        }
        pack();
        setLocationRelativeTo(null);
        setResizable(false);
        setVisible(true);
    }

    public static void main(final String[] args) {
        EventQueue.invokeLater(new Runnable() {
            public void run() {
                try {
                    new ThreadBeispiel("Thread-Beispiel");
                } catch (Exception e) {
                    e.printStackTrace();
                }
            }
        });
    }
}
\end{lstlisting}

Die Anwendung erzeugt ein oder mehrere Objekte der selbst geschriebenen
Thread-Klasse. Anschließend werden die Threads durch den Aufruf der Methode
\lstinline|start()| gestartet.

\textbf{Fragen:}

\begin{compactenum}[1.]
\item Wie viele Threads besitzt dieses Beispiel-Programm?
\item Wieso darf man die Thread-Methode \lstinline|run()| nicht direkt aufrufen?
\end{compactenum}

\end{compactenum}


\section{Animation steuern mit einem eigenen Timer-Thread}

Wenn man eine Animation programmieren möchte, muss in regelmäßigen Zeitabständen
die Methode \lstinline|repaint()| aufgerufen werden. Dazu legt man einen eigenen
Thread an, dessen einzige Aufgabe es ist die Zeit zu zählen und immer wieder mit
\lstinline|repaint()| das Neu-Zeichnen des Frames und seiner Komponenten
anzustoßen. Natürlich ist es nicht nötig, solch einen Timer selbst zu
programmieren: Wir haben ja auch bislang schon immer die Klasse
\myClass{javax.swing.Timer} für diesen Zweck verwendet. Aber es ist ein schönes
Beispiel.

\subsection{Beispiel}

\begin{lstlisting}
public class MyTimer extends Thread {
    private int repaintMillisec;
    private JFrame frame;
    private boolean threadBeendet = false;

    MyTimer(int millisec, JFrame app) {
        repaintMillisec = millisec;
        frame = app;
    }

    public void beenden() {
        threadBeendet = true;
    }

    @Override	
    public void run() {
        while (!threadBeendet) {
            try {
                frame.repaint();
                Thread.sleep(repaintMillisec);
            }
            catch (Exception e) {
            }
        }
    }
}
\end{lstlisting}

Und die Fenster-Klasse, die den Timer benutzt:

\begin{lstlisting}
æimport java.awt.*;
import java.awt.event.*;
import javax.swing.*;
import javax.swing.border.EmptyBorder;

public class SpringendeBaelle extends JFrame {
    private static final int WIDTH = 500;
    private static final int HEIGHT = 500;
    private static final Color BACKGROUND = Color.WHITE;
    private JPanel zeichenflaeche;
    private Ball b1, b2, b3;

    public SpringendeBaelle(final String title) {
        super(title);
        setDefaultCloseOperation(JFrame.EXIT_ON_CLOSE);
        JPanel contentPane = new JPanel();
        contentPane.setBorder(new EmptyBorder(5, 5, 5, 5));
        contentPane.setLayout(new BorderLayout(0, 0));
        setContentPane(contentPane);
        zeichenflaeche = new JPanel() {
            @Override
            protected void paintComponent(Graphics g) {
                super.paintComponent(g);
                myPaint(g);
            }
        };
        zeichenflaeche.setPreferredSize(new Dimension(WIDTH, HEIGHT));
        zeichenflaeche.setBackground(BACKGROUND);
        contentPane.add(zeichenflaeche);
        pack();
        setLocationRelativeTo(null);
        setResizable(false);
        setVisible(true);
        b1 = new Ball(100, 100, 50, -10, Color.RED, false);
        b2 = new Ball(200, 200, 150, -5, Color.GREEN, true);
        b3 = new Ball(400, 50, 20, 7, Color.BLUE, true);
æ        MyTimer timer = new MyTimer(30, this);
        timer.start();
æ    }
	
    public void myPaint(Graphics g) {
        // wird aufgerufen, wenn das Fenster neu gezeichnet wird
        b1.zeichnen(g);
        b2.zeichnen(g);
        b3.zeichnen(g);
    }

    public static void main(final String[] args) {
        EventQueue.invokeLater(new Runnable() {
            public void run() {
                try {
                    new SpringendeBaelle("Springende Bälle");
                } catch (Exception e) {
                    e.printStackTrace();
                }
            }
        });
    }
}
\end{lstlisting} 