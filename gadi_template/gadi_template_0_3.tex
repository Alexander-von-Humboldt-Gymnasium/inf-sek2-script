%%----------------------------------------------------------------------
% (c) 2008 by
%   Raoul Rubien - rubienr-at-sbox-dot-tugraz-dot-at
%   Thomas Quaritsch - tq-qt-sbox-dot-tugraz-dot-at
%
% Template Version 0.36 (April 2008)
%%----------------------------------------------------------------------
\newcommand{\templateVersion}    {0.36/Apr. 2008}     % template version

%%----------------------------------------------------------------------
%
% ToDo: Adjust this details
% Switch to non draft mode when your document is ready. Note that images
% are not shown while \documentclass - draft is on, wheras overfull
% boxes are.
%
%\documentclass[a4paper, 11pt, fleqn, parskip, draft]{scrartcl}
\documentclass[a4paper, 11pt, fleqn, parskip]{scrartcl}
\newcommand{\isDraft}          {true} % true | false

%%----------------------------------------------------------------------
%
% Used packages
%
\usepackage[utf8]{inputenc}           % encoding
\usepackage[ngerman] {babel}          % german wordwrap, use it for german texts
%\usepackage[english] {babel}          % english wordwrap, use it for english texts
\usepackage{algorithmic,algorithm}    % expresions
\usepackage{ifthen}                   % controllflow
\usepackage{url}                      % url
%\usepackage{hyperref}                 % href, or cross referencing throgh the document
\usepackage{color}                    % use color features
\usepackage{graphicx}                 % print pics
%\usepackage{amsmath,amssymb,amstext} % printing math formula etc.
%\usepackage{makeidx}                 % indexing
\usepackage{fancyhdr}                 % header and footer
\pagestyle {fancy}                    %
%\usepackage{listings}                 % sourcecode
\usepackage{longtable}                % a better table package
\usepackage{helvet}                   % use Helvetica (normally sans serif)
\usepackage{paralist}                 % smaller itemize vertical distance
\usepackage[includehead,includefoot]{geometry}
\usepackage{setspace}

%%----------------------------------------------------------------------
%
% some vars
%
\newcommand{\shortCourseName}    {GADI}             % abbr. course name
\newcommand{\longCourseName}     {Gesellschaftliche Aspekte der Informationstechnologie} % long course name
%\newcommand{\institute}         {Graz.TeachCenter} % institute name
%\newcommand{\instituteUrl}      {{\url{http://tugtc.tugraz.at/wbtmaster/welcome.html}}} % institute url
\renewcommand{\date}{GAdI SS 2008}


%%----------------------------------------------------------------------
%
% author details
% ToDo: Adjust this details
%
\newcommand{\Author}             {Max Mustermann}            % author details
%\newcommand{\AuthorMatnum}       {0123456}                   % author's matricula number
%\newcommand{\AuthorEmail}       {max@sbox.tugraz.at}        % author's email address
\newcommand{\course} {Gesellschaftliche Aspekte der Informationstechnologie} %
\newcommand{\method}             {Scientific Writer}         % mode is shown within top header
\newcommand{\lang}               {ger}                       % language (ger|eng)
\newcommand{\toc}                {false}                     % table of contents before \tof (true | false)
\newcommand{\tof}                {false}                     % table of figures at the very last page (true | false)


%%----------------------------------------------------------------------
%
% other usefull crubbs
% Using \sloppy may help you if your text causes overfull boxes.
%
\sloppy                                                     % less fussy about line breaking and can prevent overfull boxes
\pagenumbering{arabic}                                      % set pagenumbering to arabic
\renewcommand{\familydefault}{\sfdefault}                   % definitely sans serif


%%----------------------------------------------------------------------
%
% useful color definitions
%
\definecolor{white}              {rgb}{1,1,1}                % R:255 G:255 B:255
\definecolor{blue}               {rgb}{0.211,0.373,0.603}    % R: 76 G:129 B:178
\definecolor{lightblue}          {rgb}{0.3,0.5,0.7}          % R: 54 G: 96 B:154


%%----------------------------------------------------------------------
%
% margin parameters
%
\geometry{verbose,paperwidth=21cm,paperheight=29.7cm}
\geometry{tmargin=10mm,bmargin=13mm,lmargin=25mm,rmargin=25mm}

%%----------------------------------------------------------------------
%
% header / footer
%
\renewcommand{\headrule}{} %suppress default headrule

\lhead[]{\normalfont\sffamily\hspace*{-2.37cm}\textcolor{white}{\colorbox{lightblue}{\makebox[2.07cm][r]{\thepage}}}\hspace{0.07cm}
\course\ -- \method}
\chead[]{}
\rhead[]{}

\lfoot[]{\normalfont\sffamily\date}
\cfoot[]{}
\rfoot[]{}

%%----------------------------------------------------------------------
%
%
%
\usepackage[
    pdftex=true,             %% sets up hyperref for use with the pdftex program
    plainpages=false,        %% set it to false, if pdflatex complains: 
                             %%``destination with same identifier already exists''
    backref,                 %% adds a backlink text to the end of each biblio item
    pagebackref=false,       %% if true, creates backward references as a list of 
                             %% page numbers in the bibliography
    colorlinks=true,         %% turn on colored links (true is better for on-screen
                             %% reading, false is better for printout versions)
    bookmarks=true,          %% if true, generate PDF bookmarks (requires two passes 
                             %% of pdflatex)
    bookmarksopen=false,     %% if true, show all PDF bookmarks expanded
    bookmarksnumbered=false, %% if true, add the section numbers to the bookmarks
    %pdfstartpage={1},       %% determines, on which page the PDF file is opened
    pdfpagemode=None,        %% None, UseOutlines (=show bookmarks), UseThumbs 
                             %% (show thumbnails), FullScreen
    linkcolor=black, 
    urlcolor=black,
    citecolor=black,
    anchorcolor=black,
    pagecolor=black
  ]{hyperref}

%%----------------------------------------------------------------------
%
% section styles
%
\makeatletter
\renewcommand{\section}{                  % the style
  \@startsection {section}                % the name
  {1}                                     % the level
  {0mm}                                   % the indent
  {-1.6\baselineskip}                     % the before skip
  {0.1\baselineskip}                      % the after skip
  {\rm\Large\bfseries\textcolor{blue}} % fontstyle
}

\renewcommand{\subsection}{               % subsection style
  \@startsection {subsection}             % the name
  {2}                                     % the level
  {0mm}                                   % the indent
  {-1.6\baselineskip}                     % the before skip
  {0.1\baselineskip}                      % the after skip
  {\rmfamily\bfseries\Large\textcolor{lightblue}} % fontstyle
}

\renewcommand{\subsubsection}{             % subsection style
  \@startsection {subsubsection}           % the name
  {3}                                      % the level
  {5mm}                                    % the indent
  {-0.8\baselineskip}                      % the before skip
  {0.1\baselineskip}                       % the after skip
  {\normalfont\normalsize\sffamily\textbf} % fontstyle
}
\makeatother


%%----------------------------------------------------------------------
% some helper
%%----------------------------------------------------------------------
% insert highlighted TODOs
% \toDo {text}
\newcommand{\toDo}[1]
{
  \textcolor{red}{#1}
}

%%----------------------------------------------------------------------
%
% include a picture in the text(paced in .img/...)
%
% \includeImg {label} {width} {title} {path/to/image_without_extension(png|ps|jpg)}
\newcommand{\includeImg}[4]
{
  \begin{figure}[htbp]
    \centering
    \label{#1}
    \fbox{
      \includegraphics[width=#2\textwidth]{./img/#4}
    }
    \caption{#3}
  \end{figure}
}


%%----------------------------------------------------------------------
%
% distinguish gernan and english content
%
% \selectByLang {german content} {englich content}
\newcommand{\selectByLang}[2]{
  \ifthenelse{\equal{\lang}{ger}} {#1} {}
  \ifthenelse{\equal{\lang}{eng}} {#2} {}
}


%%----------------------------------------------------------------------
%
% standard titles
%
\newcommand{\gerMainTitle}         {Titel (deutsch) – bitte auf zwei Zeilen \\beschränken}
\newcommand{\gerSummary}           {Zusammenfassung}
\newcommand{\gerKeywords}          {Schlüsselwörter}
\newcommand{\gerIntroduction}      {Einleitung}
\newcommand{\gerCharacterization}  {Darstellung}
\newcommand{\gerDiscussion}        {Diskussion}
\newcommand{\gerBibliography}      {Literaturverzeichnis}

\newcommand{\engMainTitle}         {Title (in English) – no more than two lines at the maximum}
\newcommand{\engSummary}           {Abstract}
\newcommand{\engKeywords}          {Keywords}
\newcommand{\engIntroduction}      {Introduction}
\newcommand{\engCharacterization}  {Characterization}
\newcommand{\engDiscussion}        {Discussion}
\newcommand{\engBibliography}      {Bibliography}


%%----------------------------------------------------------------------
%
% the document
%
\begin{document}
  %draft beacon
  \ifthenelse{\boolean{\isDraft}}{
    \hspace{2mm}\small{(Template Ver.: \templateVersion)}
  }{}

  %author details
  \begin{center}
	\vspace*{3mm}
    \large \textbf{\Author} % \\ \AuthorMatnum}
    \vspace*{10mm}
  \end{center}

 % Title, Abstract and Keywords in both languages

  \selectByLang{
    \begin{spacing}{2}\textbf{\LARGE\gerMainTitle\normalsize}\end{spacing}

    \textbf{\gerSummary}

    Eine Inhaltsangabe oder Zusammenfassung ist eine Übersicht über den wesentlichen Inhalt eines Textes, Filmes oder Ereignisses. Gebräuchliche Formen von Inhaltsangaben sind das Inhaltsverzeichnis, das Abstract und andere Formen dokumentarischer Referate. Auch die englische Bezeichnung Summary ist in wissenschaftlichen Arbeiten üblich [ca. 100 Wörter]


    \textbf{\gerKeywords}

    Schlüssel"-wört"-er auf Deutsch   Schlüssel"-wört"-er auf Deutsch   Schlüssel"-wört"-er auf Deutsch   Schlüssel"-wört"-er auf Deutsch   Schlüssel"-wört"-er auf Deutsch   [max. 5]

	~

    \begin{spacing}{2}\textbf{\LARGE\engMainTitle\normalsize}\end{spacing}

    \textbf{\engSummary}

    An abstract is a brief summary of a research article, thesis, review, conference proceeding or any in-depth analysis of a particular subject or discipline, and is often used to help the reader quickly ascertain the paper's purpose. When used, an abstract always appears at the beginning of a manuscript, acting as the point-of-entry for any given scientific paper or patent application. Abstraction and indexing services are available for a number of academic disciplines, aimed at compiling a body of literature for that particular subject. [ca. 100 Wörter]

    \textbf{\engKeywords}

    Keywords in English   Keywords in English   Keywords in English   Keywords in English   Keywords in English   [max. 5]
  }
  {
    \begin{spacing}{2}\textbf{\LARGE\engMainTitle\normalsize}\end{spacing}

    \textbf{\engSummary}

    An abstract is a brief summary of a research article, thesis, review, conference proceeding or any in-depth analysis of a particular subject or discipline, and is often used to help the reader quickly ascertain the paper's purpose. When used, an abstract always appears at the beginning of a manuscript, acting as the point-of-entry for any given scientific paper or patent application. Abstraction and indexing services are available for a number of academic disciplines, aimed at compiling a body of literature for that particular subject. [ca. 100 Wörter]

    \textbf{\engKeywords}

    Keywords in English   Keywords in English   Keywords in English   Keywords in English   Keywords in English   [max. 5]
  }

  % Introduction

  \selectByLang{\section{\gerIntroduction}}
               {\section{\engIntroduction}}
  \label{sec:introduction}
  \selectByLang{Albert Einstein (* 14. März 1879 in Ulm; † 18. April 1955 in Princeton, USA) war
ein deutscher Physiker jüdischer Herkunft, dessen Beiträge zur theoretischen
Physik maßgeblich das physikalische Weltbild veränderten (Autor, 1801).

Einsteins Hauptwerk ist die Relativitätstheorie, die das Verständnis von Raum
und Zeit revolutionierte. Im Jahr 1905 erschien seine Arbeit mit dem Titel ``Zur
Elektrodynamik bewegter Körper'', deren Inhalt heute als spezielle
Relativitätstheorie bezeichnet wird. 1916 publizierte Einstein die allgemeine
Relativitätstheorie. Auch zur Quantenphysik leistete er wesentliche Beiträge:
Für seine Erklärung des photoelektrischen Effekts, die er ebenfalls 1905
publiziert hatte, wurde ihm 1921 der Nobelpreis für Physik verliehen. Seine
theoretischen Arbeiten spielten – im Gegensatz zur populären Meinung – beim Bau
der Atombombe und der Entwicklung der Kernenergie keine bedeutende Rolle.

Albert Einstein gilt als Inbegriff des Forschers und Genies. Er nutzte jedoch
seinen erheblichen Bekanntheitsgrad auch außerhalb der naturwissenschaftlichen
Fachwelt bei seinem Einsatz für Völkerverständigung und Frieden. In diesem
Zusammenhang verstand er sich selbst als Pazifist, Sozialist und Zionist.}
               {Introduction: blafasel blabiblubb} %./content/..._(ger|eng).tex

  % Characterization

  \selectByLang {\section{\gerCharacterization}}
                {\section{\engCharacterization}}
  \label{sec:characterization}
  \selectByLang{William Jefferson „Bill“ Clinton (* 19. August 1946 in Hope, Arkansas als William Jefferson Blythe III.) war von 1993 bis 2001 der 42. Präsident der Vereinigten Staaten.
Er war der Nachfolger von George H. W. Bush und Vorgänger von George W. Bush. Er gehört der Demokratischen Partei an. Clinton ist Baptist und seit 1975 mit Hillary Clinton verheiratet, mit der er die gemeinsame Tochter Chelsea Clinton (Autor, 1801) hat. 

\includeImg {img:a_spacer} {0.4} {Spacer Image} {spacer_jpg}}
               {Characterization: bla blabi blubb} %./content/..._(ger|eng).tex

  % Discussion

  \selectByLang {\section{\gerDiscussion}} 
                {\section{\engDiscussion}}
  \label{sec:discussion}
  \selectByLang{Monk ist eine erfolgreiche US-amerikanische Krimiserie. Hauptperson ist der neurotische Privatdetektiv Adrian Monk, der in San Francisco lebt.

\subsection{Überschrift 2. Ebene}
\label{subsec:ueberschrift_zweiter_ebene_01}
Adrian Monk, verkörpert von Tony Shalhoub, war früher Polizist beim Mordderzernat des San Francisco Police Department. Als seine Ehefrau Trudy Anne bei einem Anschlag ums Leben kam, verschlechterten sich seine zahlreichen ``Macken'' enorm und wurden zu einer psychischen Störung. Er verließ drei Jahre lang seine Wohnung nicht und wurde aus dem Polizeidienst entlassen.

Zu Beginn der Serienhandlung hat sich sein Zustand gebessert, allerdings bleiben zahlreiche Phobien (Angststörungen). Er benötigt daher ständig Hilfe durch einen persönlichen Assistenten; bis in die dritte Staffel ist dies Sharona Flemming (kongenial gespielt von Bitty Schramm), eine ehemalige Krankenschwester. Ihre Aufgabe ist es vor allem, alles von Monk fernzuhalten, was ihm Angst macht - und das ist eine Menge: Er leidet unter anderem an Angst vor Höhen (Akrophobie), Enge (Klaustrophobie), Dunkelheit (Achluophobie), Berührungen (Aphephosmophobie), Bakterien (Bacteriophobie) und vielem mehr.

Außerdem kann er Unordnung nicht ertragen; ein schief hängendes Bild muss sofort gerade gerückt, asymmetrisch angeordnete Gegenstände unverzüglich an den "richtigen" Platz gestellt werden, vorher ist Monk (Autor, 1801) nicht in der Lage, sich etwas anderem zuzuwenden, selbst wenn dadurch extreme Situationen entstehen.

Monk arbeitet als Privatdetektiv und freier Berater für die Polizei in schwierigen Fällen. Dabei hat er gerade durch seine Phobien einen guten Spürsinn für Dinge, die nicht in Ordnung sind. Meist wird er von seinem früheren Vorgesetzten, Captain Leland Stottlemeyer, beauftragt, der Monks unorthodoxer Methode skeptisch gegenübersteht, letztlich aber meist keine andere Wahl hat als sich auf Monk zu verlassen.

Monks Ziel ist die Wiederaufnahme in den Polizeidienst. In einer Folge nimmt er an einem Einstellungstest teil, scheitert jedoch tragisch am Ausfüllen eines Fragebogens

\subsection{Überschrift  2. Ebene}
\label{subsec:ueberschrift_zweiter_ebene_02}
Monk ist natürlich eine fiktive Filmfigur mit bunt zusammen gemischten Charaktermerkmalen und pathologischen Merkmalen aus vielen unterschiedlichen Bereichen. Neben den zahlreichen Phobien könnten seine Inselbegabungen z.B. auch als Asperger-Syndrom gedeutet werden; letztendlich entspricht seine Ansammlung von Symptomen jedoch keinem realen Krankheitsbild.

\subsubsection{Überschrift  3. Ebene}
\label{subsec:ueberschrift_dritter_ebene_01}
\begin{compactitem}
  \item Natalie Teeger (Traylor Howard) ist Monks Assistentin, nachdem die Darstellerin Sharonas, Bitty Schramm, die Serie verließ. In der Handlung wird das so gelöst, dass Sherona ihren Ex-Mann heiratet und nach New Jersey zieht. Natalie hat weniger Verständnis für Monks Phobien als Sharona, versucht sie ihm sogar oft „auszutreiben“, was stets in einem Fiasko endet. Sie hat eine Tochter, Julie (Emmy Clarke). 
  \item Lieutenant Randall Disher (Jason Gray-Stanford) ist Stottlemeyers Assistent; seine absurd-naiven Ideen und Theorien tragen zur Belustigung bei. 
  \item Dr. Charles Kroger (Stanley Kamel) ist Monks Psychiater; obwohl ihn sein Patient oft über Gebühr beansprucht, verliert er nie die Fassung. 
  \item Trudy Anne Monk (Stellina Rusich / Melora Hardin) ist Monks verstorbene Ehefrau. Sie wurde in einem Parkhaus von einer Autobombe getötet. In Folge 51 (Mr. Monk und Mrs. Monk) taucht sie vermeintlich wieder auf. Es handelt sich jedoch nur um eine Doppelgängerin, die obendrein am Ende der Episode ums Leben kommt. 
  \item Adrians Bruder Ambrose Monk (gespielt von John Turturro) ist ebenfalls hochgradig neurotisch. Er schreibt beruflich Gebrauchsanleitungen, überwiegend für Elektrogeräte, und beherrscht daher acht Sprachen, inklusive Hochchinesisch. Der Vater der beiden, Jake Monk (Dan Hedaya), verschwand, als beide noch jung waren. Seitdem wartet Ambrose auf ihn und hat das Haus nicht verlassen. Erst als er durch ein Feuer umgebracht werden soll, überwindet er diese Phobie und wird von seinem Bruder gerettet.
\end{compactitem}

Zitieren von Literatureinträgen funktioniert so: \cite{sample1}}
               {Discussion: blubber blaflup} %./content/..._(ger|eng).tex

  % Bibliography

  \selectByLang {\section{\gerBibliography}}
                {\section{\engBibliography}}
  \label{sec:bibliography}
  \renewcommand{\refname}{}
  \selectByLang{\begin{thebibliography}{1}
\bibitem{sample1} \textbf{Nachname, X.} (Erscheinungsjahr). Titel. Erscheinungsort: Verlag.
\bibitem{sample2} \textbf{Nachname, X. \& Nachname, Y.} (Erscheinungsjahr). Titel. Erscheinungsort: Verlag.
\bibitem{sample3} \textbf{Nüesch, C., Wilbers, K. \& Zellweger, F.} (2005) Die Förderung überfachlicher Kompetenzen an der
HSG. St. Gallen: Institut für Wirtschaftspädagogik.
\bibitem{sample4} \textbf{Oblinger, D. C., \& Oblinger, J. L.} (Hrsg.). (2005). Educating the Net Generation. Boulder, CO:
Educause. \url{http://www.educause.edu/educatingthenetgen}, Stand vom 26. September 2005.
\bibitem{sample5} \textbf{Nachname, X., Nachname, Y. \& Nachname, Z.} (Erscheinungsjahr). Titel. Erscheinungsort: Verlag.
\bibitem{sample6} \textbf{Nachname, X.} (Erscheinungsjahr). Titel. Zeitschrift / Journal für xxx, Nr / Jg. xxx, S. xxx-xxx.
\end{thebibliography}
}
               {\begin{thebibliography}{1}
\bibitem{sample1} \textbf{Surname, X.} (year of publication). Title. Place of publication: Publisher.
\bibitem{sample2} \textbf{Surname, X. \& Name, Y.} (year of publication). Title. Place of publication: Publisher.
\bibitem{sample3} \textbf{Nüesch, C., Wilbers, K. \& Zellweger, F.} (2005) Die Förderung überfachlicher Kompetenzen an der
HSG. St. Gallen: Institut für Wirtschaftspädagogik.
\bibitem{sample4} \textbf{Oblinger, D. C., \& Oblinger, J. L.} (Hrsg.). (2005). Educating the Net Generation. Boulder, CO:
Educause. \url{http://www.educause.edu/educatingthenetgen}, Stand vom 26. September 2005.
\bibitem{sample5} \textbf{Surname, X., Name, Y. \& Surname, Z.} (year of publication). Titel. Place of publication: Publisher.
\bibitem{sample6} \textbf{Surname, X.} (year of publication). Title. Magazine / Journal of xxx, Nr / Year. xxx, S. xxx-xxx.
\end{thebibliography}} %./content/..._(ger|eng).tex

  %tables of contents
  \ifthenelse{\boolean{\toc}}{\newpage\tableofcontents}{}
  \ifthenelse{\boolean{\tof}}{\listoffigures}{}
\end{document}
% EoF
