\chapter{SQL -- Wiederholung}
\renewcommand{\chaptertitle}{SQL -- Wiederholung}

\lehead[]{\sf\hspace*{-2.00cm}\textcolor{white}{\colorbox{lightblue}{\parbox[c][0.70cm][b]{1.60cm}{
\makebox[1.60cm][r]{\thechapter}\\ \makebox[1.60cm][r]{ÜBUNG}}}}\hspace{0.17cm}\textcolor{lightblue}{\chaptertitle}}
\rohead[]{\textcolor{lightblue}{\chaptertitle}\sf\hspace*{0.17cm}\textcolor{white}{\colorbox{lightblue}{\parbox[c][0.70cm][b]{1.60cm}{\thechapter\\
ÜBUNG}}}\hspace{-2.00cm}}
%\chead[]{}
\rehead[]{\textcolor{lightblue}{AvHG, Inf, My}}
\lohead[]{\textcolor{lightblue}{AvHG, Inf, My}}

\section{Aufgaben zur Wiederholung}

\subsection{Aufgabe 1: Datenbankentwurf}

Für das Warenlager eines großen Möbelhauses soll eine Datenbank entworfen
werden. Ein Mitarbeiter des Möbelhauses erklärt dir, welche Informationen in der
Datenbank abgespeichert werden sollen:

\begin{quotation}
\noindent Jeder Artikel, den wir führen, besitzt eine Artikelnummer. In der
Datenbank soll jedoch auch die Bezeichnung des Artikels abgespeichert werden,
da nicht alle Mitarbeiter mit der Nummer etwas anfangen können. Außerdem muss
abgespeichert werden, wie viele Stücke eines Artikels wir zur Zeit auf Lager
haben. Für den Fall, dass ein Artikel nachbestellt werden muss, muss aus der
Datenbank ersichtlich sein, von welchem Lieferanten wir den Artikel beziehen.
Jeder Artikel wird von genau einem Lieferanten bezogen.

\noindent Die meisten Lieferanten liefern uns selbstverständlich mehrere
Artikel. Von dem Lieferanten müssen wir den Firmennamen, die Telefonnummer und
die Adresse wissen.

\noindent Die Datenbank soll auch eine Kundenverwaltung enthalten. Alle Kunden
sollen mit Vorname, Nachname und Adresse gespeichert werden. Falls ein Kunde eine
Bestellung vornimmt, soll aus der Datenbank ersichtlich sein, wie viele Stücke
eines Artikels er bestellt hat, wann die Bestellung aufgegeben wurde und welchen
Status die Bestellung hat (erfolgreich geliefert, noch offen, konnte nicht
geliefert werden).
\end{quotation}

\begin{compactenum}[a)]
\item Entwirf ein ER-Diagramm für die Datenbank. Die Adresse eines Kunden und
eines Lieferanten braucht nicht detailliert in einzelne Attribute unterteilt
werden.
\item Berücksichtige, dass verschiedene Kunden mit der gleichen Adresse
existieren können. Musst du deinen Datenbankentwurf korrigieren, um die
Bedingungen der dritten Normalform zu erfüllen?
\end{compactenum}


\subsection{Aufgabe 2: Normalformen}

Erkläre die Bedeutung der drei Normalformen, in denen sich eine Datenbank befinden kann.


\subsection{Aufgabe 3: Urlaubs-Datenbank}

In einer großen Firma beschließen einige Mitarbeiter, ihren nächsten
Sommerurlaub gemeinsam zu verbringen. Um alle Wünsche unter einen Hut bringen zu
können, legt Helga Schneider eine Datenbank an.

In der Datenbank befinden sich vier verschiedene Vorschläge für Reiseziele.
Jeder Mitarbeiter konnte angeben, welche der vier Reisen er mitmachen würde.
Außerdem sollte jeder Mitarbeiter die Zeiten angeben, zu denen er an der Reise
teilnehmen kann.

\begin{compactenum}[a)]

\item Im Kurs-Repository findest du die Datei \myFile{urlaubsdatenbank.sql}.
Erzeuge mit Hilfe dieser Datei eine neue Datenbank auf deinem Computer.
Analysiere die Datenbank mit dem Tool MySQL-Workbench und zeichne ein
ER-Diagramm, das die Struktur der Datenbank abbildet.

\item Die Datenbank befindet sich in der ersten Normalform. Begründe, wieso die
 erste Normalform erfüllt wird, und wieso die zweite Normalform nicht erfüllt wird.

\item Nimm die nötigen Änderungen vor, um die Bedingungen der dritten
Normalform zu erfüllen und speichere das geänderte SQL-Skript ab.

\item Erzeuge SQL-Kommandos, die die nachfolgenden Aufgaben lösen (basierend
auf dem von dir geänderten SQL-Skript der Urlaubs-Datenbank), und speichere sie
in einer Datei.

\begin{compactenum}[1.]
\item Ermittle, wie viele Mitarbeiter in der Datenbank eingetragen sind.

\item Erstelle eine Liste, die angibt welche Mitarbeiter (Vorname, Nachname)
sich für welches Reiseziel (Land) interessieren. Sortiere die Liste aufsteigend
nach dem Land.

\item Zähle die Anzahl der Mitarbeiter, die sich für die einzelnen Reisen
interessieren. Das Ergebnis soll eine Tabelle sein mit der Reise (Land und
Beschreibung) und einer Spalte, die die Anzahl der Interessenten angibt.

\item Ermittle diejenigen Mitarbeiter mit Vorname und Nachname, die noch keine
Urlaubszeiten angegeben haben.

\item Ermittle alle Mitarbeiter (Vorname und Nachname), die mit Helga Schneider
in der selben Abteilung arbeiten. Helga Schneider darf nicht in der Liste
erscheinen.

\item Liste die Namen (Vorname, Nachname) aller Mitarbeiter auf, die gerne einen
Badeurlaub machen würden (\glqq Badeurlaub\grqq\ steht in der Spalte
Beschreibung der Tabelle Reiseziel). In der Liste darf kein Name doppelt
aufgeführt werden.

\item Zähle die Anzahl der Mitarbeiter, die gerne nach Spanien fahren würden.
Das Ergebnis ist eine einzige Zahl.

\item Liste alle Mitarbeiter mit Vor- und Nachname auf, die sich für mehr als
zwei Reiseziele interessieren.

\item Ermittle alle Mitarbeiter, die mit Kai Schneider ein Reiseziel gemeinsam
haben, mit Vornamen und Nachnamen. Kai Schneider darf auch in der Liste
erscheinen. Es dürfen jedoch keine Namen doppelt aufgelistet werden.
\end{compactenum}
\end{compactenum}