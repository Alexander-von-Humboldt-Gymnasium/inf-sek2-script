\section{Software}

Im Folgenden wird die Software von PCs betrachtet.

\subsection{BIOS}

Startet man den Rechner (\glqq hochfahren\grqq , booten), wird, bevor das
Betriebssystem in Aktion tritt, das sogenannte \emph{BIOS} (Basic Input Output
System) gestartet. Das BIOS ist ein Ein- und Ausgabesystem, welches unabhängig
vom nachfolgend installierten Betriebssystem vom Hersteller mitgeliefert wird.

Das BIOS steht schreibgeschützt in einem Speicherbaustein auf dem Mainboard
(ROM-Baustein, engl. für Read Only Memory). Es testet bei jedem Computerstart
alle Hardwarekomponenten, startet den Rechner und lädt das Betriebssystem von
der Festplatte. Während des laufenden Betriebs kann das BIOS elementare
Aufgaben zur Ansteuerung von Hardwarekomponenten wie Tastatur, Maus, Festplatte
und Grafikkarte übernehmen und arbeitet damit dem Betriebssystem
zu.\footnote{Moderne Betriebssysteme nutzen diese Funktionalität allerdings nur
noch während der ersten Phase des Bootens. Denn dort gibt es ein
klassisches Henne-Ei-Problem: Zum Booten muss das Betriebssystem
Daten von der Festplatte laden und gegebenenfalls auch auf
Tastatureingaben des Benutzers reagieren können. Ebenso sollen
Ausgaben auf dem Bildschirm erscheinen. Aber das Betriebssystem
läuft ja noch gar nicht \ldots Anschließend werden diese BIOS-Funktionen dann
nicht mehr benutzt.}

\subsection{Betriebssystem}

Das Betriebssystem ist eine spezielle Software zur Steuerung der
Grundfunktionen eines Computers und zur Verwaltung der Hardware. Es besteht aus
einer Reihe von Systemprogrammen (Systemsoftware), die den Umgang mit CPU,
Hardwarekomponenten und Anwendungsprogrammen ermöglichen, und ist damit die
Schnittstelle zwischen Mensch und Computer sowie zwischen Anwendungsprogrammen
und Hardware.

\subsubsection{Aufgaben eines Betriebssystems}

Das Betriebssystem steuert alle zum Betrieb des Rechners notwendigen
Verwaltungsaufgaben. Ein Betriebssystem gehört in der Regel zur
Grundausstattung des Computers. Es ist jedoch prinzipiell austauschbar. Die
wichtigsten Aufgaben des Betriebssystems sind:

\begin{compactitem}
\item[\textbf{Speicherverwaltung}]
Verwaltung des Hauptspeichers
\item[\textbf{Dateiverwaltung}]
Organisation der Daten auf Festplatten und anderen Datenträgern. Dazu gehört z.B. die
Vorbereitung der Datenträger (Partitionieren von Festplatten, Formatieren und Benennen von
Datenträgern), die Erstellung von Verzeichnissen, das Kopieren, Löschen, Verschieben und
Umbenennen von Dateien.
\item[\textbf{Programmverwaltung} (Fachwort: \emph{Prozessverwaltung})]
Ausführung von Anwendungsprogrammen und Aufteilung der Rechenzeit der CPU zwischen allen
momentan laufenden Programmen.
\item[\textbf{Geräteverwaltung}]
Steuerung der Arbeit aller angeschlossenen internen und externen Geräte.
\end{compactitem}

\subsubsection{Bestandteile eines Betriebssystems}

Betriebssysteme vereinen sogenannte residente und transiente Bestandteile:

Residente Bestandteile werden nach dem Start sofort in den Arbeitsspeicher
(RAM) geladen und verbleiben dort während der gesamten Betriebszeit des
Computers. Diese Bestandteile ermöglichen die Kommunikation zwischen Nutzer und
Rechner sowie das Laden und Starten von Anwendungsprogrammen.

Transiente Bestandteile befinden sich in eigenständigen Dateien, die auf der
Festplatte in einem gesonderten Verzeichnis gespeichert sind und nur in den
Arbeitsspeicher geladen werden, wenn es notwendig ist.

\subsubsection{Grafische Benutzeroberfläche}

Die Benutzeroberfläche ist der Teil eines Programms, den der Benutzer auf dem
Bildschirm sehen kann. Sie bildet die Schnittstelle zwischen dem Programm und
seinem Benutzer, welcher über die Tastatur oder die Maus Dienste anfordern
kann.

Die einfachste Version einer Benutzeroberfläche zeigt eine Kommandozeile. Der
Benutzer hat die Möglichkeit zeilenweise Text einzugeben und erhält als Antwort
vom Programm wieder eine Textausgabe. Moderne Programme und Betriebssysteme
haben eine sogenannte grafische Benutzeroberfläche, die dem Benutzer ein
Fenstersystem mit vielen bunten Bildern präsentiert.

Die Benutzeroberfläche des Betriebssystems eines PCs ist der sogenannte
\emph{Desktop} (deutsche Übersetzung: \glqq Schreibtisch\grqq ).


\subsection{Treiber}

Treiber sind Programme zur Ansteuerung einer Software- oder
Hardware-Komponente. Ein Treiber ermöglicht einem Anwendungsprogramm die
Benutzung einer Komponente, ohne deren detaillierten Aufbau zu kennen.

Zum Beispiel kann ein Textverarbeitungsprogramm nicht im Voraus alle
verschiedenen Drucker kennen, die es vielleicht einmal bedienen soll. Wenn der
Benutzer aus dem Datei-Menü den Befehl „Drucken“ auswählt, soll der Druck
funktionieren, egal welcher Drucker angeschlossen ist. Die Druckbefehle des
Textverarbeitungsprogramms (z.B. \glqq Drucke ein kursives 'a'\grqq , \glqq Neue
Zeile anfangen\grqq\ oder \glqq Seitenumbruch\grqq ) werden deshalb vom
Betriebssystem an ein sogenanntes Treiberprogramm (kurz Treiber) weitergeleitet, das vom Hersteller
des ausgewählten Druckers geschrieben wurde. Das Treiberprogramm weiß genau,
wie der spezielle Drucker angesteuert werden soll und übersetzt die Befehle der
Textverarbeitung in geeignete Drucker-Befehle.

Treiber für gängige Geräte sind meist schon im Betriebssystem vorhanden. Wenn
aber ein neues Gerät, das dem Betriebssystem noch nicht bekannt ist, an einen
Rechner angeschlossen wird, so muss auch ein zugehöriges Treiberprogramm
installiert werden.


\subsection{Anwendungssoftware}

Die Art der Aufgaben, die ein Computer erledigen kann, ist nicht von
vornherein festgelegt. Für den Computer können beliebig viele
Anwendungsprogramme generiert werden. Auf einem PC können mehrere
Anwendungsprogramme parallel laufen. Ein ausführbares Programm besteht aus
einem für den Rechner zugeschnittenen Maschinencode und kann daher immer nur
auf einem bestimmten Betriebssystem laufen. Unter Windows werden ausführbare
Programme mit der Endung \myFile{*.exe} gekennzeichnet. \glqq exe\grqq\ steht
dabei für das englische Wort \emph{executable} (auf deutsch: ausführbar).

Typische Aufgaben für Anwendungsprogramme sind:

Textverarbeitung, Tabellenkalkulation, Buchhaltung, Spiele, Internet Browser,
usw.