\chapter{SQL -- Komplexes Beispiel}
\renewcommand{\chaptertitle}{SQL -- Komplexes Beispiel}

\lehead[]{\sf\hspace*{-2.00cm}\textcolor{white}{\colorbox{lightblue}{\parbox[c][0.70cm][b]{1.60cm}{
\makebox[1.60cm][r]{\thechapter}\\ \makebox[1.60cm][r]{ÜBUNG}}}}\hspace{0.17cm}\textcolor{lightblue}{\chaptertitle}}
\rohead[]{\textcolor{lightblue}{\chaptertitle}\sf\hspace*{0.17cm}\textcolor{white}{\colorbox{lightblue}{\parbox[c][0.70cm][b]{1.60cm}{\thechapter\\
ÜBUNG}}}\hspace{-2.00cm}}
%\chead[]{}
\rehead[]{\textcolor{lightblue}{AvHG, Inf, My}}
\lohead[]{\textcolor{lightblue}{AvHG, Inf, My}}

\section{Firmen-Datenbank}

\subsection{Aufgabe 1: Datenbank Firma untersuchen}

Im Kurs-Repository liegt die Datei \myFile{firma.sql}.
Erzeuge mit Hilfe dieser Datei die Datenbank \myUserInput{firma} und untersuche
die neue Datenbank.

Erstelle ein UML-Klassendiagramm, das die Tabellen und ihre Beziehungen
zueinander beschreibt. Markiere im Klassendiagramm alle Datenfelder, die zum
Primärschlüssel (PK) oder Fremdschlüssel (FK) einer Tabelle gehören.


\subsection{Aufgabe 2: Datenbankabfragen}

Führe auf der Firmen-Datenbank die folgenden Abfragen durch:

\begin{compactenum}[a)] 
\item Erstelle eine Liste aller Abteilungen (Abteilungsname und
Abteilungsnummer) und der Namen der zugehörigen Abteilungsleiter (Vor- und
Nachname). Sortiere die Liste absteigend nach dem Abteilungsnamen.

\item Erstelle eine Liste aller Angestellten und ihrer Angehörigen. Es sollen
der Vor- und Nachname der Angestellten sowie der Name, das Geschlecht und der
Verwandtschaftsgrad der Angehörigen aufgelistet werden. Auch Angestellte, die
keine Angehörigen besitzen, sollen in der Tabelle aufgelistet werden.
Sortiere die Liste alphabetisch zuerst nach dem Nachnamen und anschließend nach
dem Vornamen der Angestellten. Danach soll die Liste nach dem Namen der
Angehörigen sortiert werden.

\item Ermittle den Namen (Vor- und Nachname) des Vorgesetzten von Jennifer
Wallace.

\item Zähle die Anzahl der Projekte, die momentan in der Firma durchgeführt
werden.

\item Erstelle eine Liste mit allen Angestellten (Vor- und Nachname) und den
Projekten (Projekt-Name und Projekt-Nummer), an denen sie arbeiten.

\item Erstelle eine Liste aller Mitarbeiter (Vorname, Nachname, Geburtstag,
Geschlecht), die am Projekt „Newbenefits“ arbeiten.

\item Ermittle den Abteilungsleiter (Vor- und Nachname), der für das Projekt
„Reorganization“ zuständig ist.

\item Zähle die Anzahl der Mitarbeiter im Projekt „Reorganization“.

\item Liste alle Orte auf, in denen die Firma eine Niederlassung besitzt. Jeder
Ort soll nur einmal aufgelistet werden.

\item Zähle die Anzahl der Projekte, an denen jeder einzelne Mitarbeiter
arbeitet. Die Mitarbeiter sollen mit Vor- und Nachnamen aufgelistet werden.

\item Liste alle Projekte mit Namen auf, die am selben Ort wie das Projekt
„Newbenefits“ stattfinden. Das Projekt „Newbenefits“ selbst soll nicht in der
Liste erscheinen.

\item Liste alle Mitarbeiter mit Vornamen, Nachnamen und Angestellten-Nummer
auf, die mindestens drei Angehörige besitzen.
\end{compactenum}