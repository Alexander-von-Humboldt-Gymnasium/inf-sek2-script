\section{UML}

\subsection{Aufgabe 1: UML-Klassendiagramme}

Beschreibe die Beziehungen zwischen den Klassen in einem UML-Klassendiagramm
(soweit angebracht mit Beziehungsname und Multiplizität):

\begin{compactenum}
\item Hund, Schwanz, Mensch, Dackel
\item Kanarienvogel, Käfig, Gitterstab
\item Bürgerweide, Fahrgeschäft, Karussell, Achterbahn, Sitzplatz,
Mensch
\end{compactenum}


\subsection{Aufgabe 2: Model-Agentur}

Eine Model-Agentur hat dich beauftragt, ein Computersystem zu entwickeln.
Erstelle ein UML-Klassendiagramm, das die Informationen, die dir die Chefin der
Agentur gibt, geeignet abbildet:

\begin{quotation}
\noindent Für unsere Agentur arbeiten weibliche und männliche Models. Von jedem
Model speichern wir die Adresse, die Größe, das Gewicht, die Haarfarbe und ein
Foto. „Normale“ Models werden von uns für jeden Auftrag einzeln bezahlt. Mit
Models, die besonders erfolgreich sind, machen wir jedoch häufig Exklusiv-
Verträge, in denen vereinbart wird, dass sie ausschließlich für unsere Agentur
arbeiten dürfen. Zum Exklusiv-Vertrag gehört auch ein monatliches Grundgehalt,
das wir an das Model zahlen. Zusätzlich zum Grundgehalt erhalten auch die
Models mit Exklusiv-Verträgen für jeden Auftrag eine Bezahlung.

Wir verwalten natürlich auch eine Liste von Kunden (wie z.B. Modehäuser), die
regelmäßig bei uns Models anfordern. Für jeden Kunden arbeitet mindestens ein
Model. Die Models dagegen können für beliebig viele Kunden arbeiten. Weniger
begehrte Models werden häufig auch gar nicht angefordert.

Von den Kunden speichern wir die Adresse und die Anzahl unserer Models, die
schon für diesen Kunden gearbeitet haben. Wir verwalten auch für jeden Kunden
eine Liste von Großereignissen (wie z.B. Modeschauen), bei denen Models benötigt
werden. Für die Großereignisse speichern wir die Art der Veranstaltung und die
Anzahl der benötigten Models. Die Kunden veranstalten zwischen null und fünf
Großereignisse im Jahr.
\end{quotation}

Anmerkung: Die Adresse braucht nicht in einzelne Attribute aufgegliedert zu
werden.


\subsection{Aufgabe 3: Weihnachtsmann-Roboter}

Zu Weihnachten verkleiden sich häufig Vater, Onkel oder freundliche Bekannte als
Weihnachtsmann, um den Kindern die Geschenke zu bringen. Jetzt hat eine Firma
für diese Zwecke einen Weihnachtsmann-Roboter entwickelt. Eine Anwendung, die
in den kommenden Jahren sicher große Erfolge feiern wird.

Schreibe ein Zustandsdiagramm für den im folgenden Text beschriebenen
Weihnachtsmann-Roboter:

\begin{quotation}
\noindent \textbf{Weihnachtsmann-Roboter (Preiswerte Version für nur 1 Kind)}

\noindent Der Weihnachtsmann-Roboter wartet geduldig (ohne zu murren auch
stundenlang!) draußen vor der Tür, bis ihm ein Mitglied der Familie die Tür
öffnet. Nachdem er das Wohnzimmer betreten hat, sagt er zunächst ein Gedicht
auf („Draußen vom Walde komm ich her..“). Danach wendet er sich an das Kind und
stellt ihm eine Quizfrage (die Frage „Warst du auch schön brav“ ist heutzutage
schließlich total veraltet). Wenn das Kind die Quizfrage korrekt beantwortet,
erhält es ein großes Geschenk. Falls es eine falsche Antwort gibt, darf es noch
ein zweites Mal versuchen, die Frage zu beantworten. Wenn die zweite Antwort
richtig war, erhält das Kind ebenfalls das große Geschenk. Andernfalls droht der
Weihnachtsmann ihm – nach alter Tradition – mit der Rute und überreicht ihm
dann ein kleines Geschenk. Zum Schluss singt der Weihnachtsmann der Familie
noch ein Weihnachtslied vor und verlässt dann das Haus.
\end{quotation}


\subsection{Aufgabe 4: Weihnachts-Lieferservice}

Der Weihnachts-Lieferservice muss besonders gut organisiert werden, da jedes
Jahr zu Weihnachten Millionen von Kunden zu beliefern sind. Deshalb möchte
jetzt auch der Weihnachtsmann modernisieren und sein Büro „auf Computer
umstellen“. Du hast den Auftrag, das Computersystem für den Weihnachtsmann zu
entwickeln. Schreibe ein Klassendiagramm, das die Abläufe im
Weihnachts-Lieferservice darstellt:

\begin{quotation}
\noindent Die Annahme, das es nur einen einzigen Weihnachtsmann gibt, ist
natürlich eine kindliche Illusion. Ein Mann alleine könnte ja nie die ganze
Arbeit bewältigen, die an Heiligabend anfällt. Es gibt hunderte von
Weihnachtsmännern und -frauen, die jeweils für einen bestimmten Bezirk
zuständig sind. Die Weihnachts“männer“ unterscheiden sich durch verschiedene
Vornamen, das Geschlecht und ihr Alter voneinander. Jeder Weihnachtsmann (und
jede Weihnachtsfrau) besitzt vier Rentiere, die er während der Sommermonate
hart trainiert, damit sie für die großen Anforderungen am Weihnachtsabend fit
sind. Jedes Rentier hat seinen eigenen Namen. Für die Bestellannahme ist eine
große Schar von Engeln zuständig. 15 bis 30 Engel arbeiten jeweils als
Hilfskräfte für einen Weihnachtsmann. Einfache Engel können nur normale
Bestellungen annehmen. Oberengel können auch Fern- Aufträge für Lieferungen in
fremde Länder bearbeiten. Die eigentliche Auslieferung der Geschenke erfolgt
jedoch einzig und allein durch die Weihnachtsmänner.
\end{quotation}