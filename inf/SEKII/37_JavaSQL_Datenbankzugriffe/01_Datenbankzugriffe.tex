\chapter{Datenbankzugriffe mit Java}
\renewcommand{\chaptertitle}{Datenbankzugriffe mit Java}

\lehead[]{\normalfont\sffamily\hspace*{-2.00cm}\textcolor{white}{\colorbox{lightblue}{\makebox[1.60cm][r]{\thechapter}}}\hspace{0.17cm}\textcolor{lightblue}{\chaptertitle}}
\rohead[]{\textcolor{lightblue}{\chaptertitle}\normalfont\sffamily\hspace*{0.17cm}\textcolor{white}{\colorbox{lightblue}{\makebox[1.60cm][l]{\thechapter}}}\hspace{-2.00cm}}
%\chead[]{}
\rehead[]{\textcolor{lightblue}{AvHG, Inf, My}}
\lohead[]{\textcolor{lightblue}{AvHG, Inf, My}}

\lstset{style=myJava}

Um mit einer Datenbank zu arbeiten, braucht man neben dem Datenbank-Server auch
einen Client. Bisher hast du immer das SQL Explorer Plugin in Eclipse benutzt um 
mit der Datenbank zu arbeiten. Aber es gibt Alternativen. Etwa den einfachen
MySQL-Kommandozeilen-Client oder auch das Tool MySQL Workbench.

Neben diesen und anderen verfügbaren Clients, kann man aber auch einen eigenen
Client programmieren. Wenn du etwa im Reisebüro bist, läuft dort im Hintergrund
sicher eine Datenbank. Auf diese wird über einen speziell für diese Aufgabe
entwickelten Client zugegriffen. Solche Clients können Web-basiert sein (also
im Browser laufen), oder auch ein normales Anwendungsprogramm. So wie man es
etwa mit Java entwickeln könnte.

\section{SQL-Package}

Um Datenbankzugriffe mit Java ausführen zu können muss man das Package
\myPackage{java.sql} importieren:

\begin{lstlisting}
import java.sql.*;
\end{lstlisting}

Alle nachfolgend beschriebenen Java-Befehle werden grundsätzlich mit einem
\myCmd{try}-\myCmd{catch}-Block überwacht, weil beim Auftreten von Fehlern
Exceptions geworfen werden.

\section{Verbindung zur Datenbank aufbauen}

Zunächst muss die Verbindung zur der gewünschten Datenbank (im Beispiel
\myUserInput{MeineDatenbank}) hergestellt werden. Dabei muss auch der Name des
Benutzers angegeben werden, der die Datenbank verwenden möchte. Wenn die
Verbindung erfolgreich aufgebaut wurde, wird ein Objekt vom Typ
\myClass{Connection} zurückgegeben.

\begin{lstlisting}
Connection conn = DriverManager.getConnection(
        "jdbc:mysql://localhost/MeineDatenbank?serverTimezone=UTC&useSSL=false",
        "root", "root");
\end{lstlisting}

Um Kommandos an die Datenbank schicken zu können, muss man sich anschließend vom
\myClass{Connection}-Objekt ein Objekt vom Typ \myClass{Statement} besorgen.

\begin{lstlisting}
Statement stmt = conn.createStatement();
\end{lstlisting}

Sowohl \myClass{Connection}- als auch \myClass{Statement}-Objekte sind
Ressourcen, die man nach der Benutzung wieder frei geben sollte. Zu diesem Zweck
bieten beide Klassen jeweils eine Methode \lstinline|close()|.

Seit Java SDK7 ist es allerdings nicht mehr nötig (aber möglich), diese
Ressourcen manuell frei zu geben. Statt dessen bietet sich das mit JDK7 neu
eingeführte \myUserInput{try-with-ressource} an (welches sich im Übrigen nicht
nur für Datenbank-Objekte, sondern auch für Datei-Objekte anbietet):

\begin{lstlisting}
try (`\slshape{Initialisierung der Ressourcen}`) {
  `\slshape{Code, in dem es zu Exceptions kommen kann}`
} catch (`\slshape{Exceptions}`) {
  `\slshape{Fehlerbehandlung}`
}
\end{lstlisting}

Wenn nicht nur eine, sondern wie in unserem Beispiel mehrere Ressource
initialisiert werden sollen, dann werden die einzelnen
Initialisierungsanweisungen durch Semikolon von einander getrennt.

Die in den Runden Klammern nach dem Schlüsselwort \myUserInput{try}
initialisierten Ressourcen werden automatisch mit dem Verlassen des
\myUserInput{try-catch}-Blocks geschlossen.

\section{Datenbankabfragen}

Datenbankabfragen kann man mit der Methode \lstinline|executeQuery()| des
\myClass{Statement}-Objekts stellen:

\begin{lstlisting}
public ResultSet executeQuery(String sql) throws SQLException
\end{lstlisting}

Als Parameter übergibt man einen String mit einer SQL-Select-Abfrage.
Zurückgegeben wird ein Objekt vom Typ \myClass{ResultSet}, das die
Ergebnis-Datensätze enthält. Beispiel:

\begin{lstlisting}
ResultSet rs = stmt.executeQuery("SELECT * FROM tier WHERE geschlecht='w'");
\end{lstlisting}

Das \myClass{ResultSet}-Objekt besitzt eine Methode \lstinline|next()|, mit der
alle Datensätze der Ergebnismenge schrittweise durchlaufen werden können:

\begin{lstlisting}
boolean next()
\end{lstlisting}

Nach dem Aufruf von \lstinline|executeQuery()| steht der Satzzeiger noch vor dem
ersten Datensatz. Durch einen Aufruf der Methode \lstinline|next()| wird der
Zeiger auf den ersten Datensatz gesetzt, sofern die Ergebnismenge nicht leer
ist. Falls ein (weiterer) Datensatz gefunden wurde, gibt \lstinline|next()|
\myCmd{true} zurück. Wenn kein weiterer Datensatz existiert, erhält man den
Rückgabewert \myCmd{false}.

Nachdem man sich mit \lstinline|next()| einen Datensatz besorgt hat, kann man
mit \lstinline|getXXX()|-Methoden, auf die Spalten des Datensatzes zugreifen. Es
gibt eine \lstinline|getXXX()|-Methode für alle verschiedenen Datentypen, z.B.:
\lstinline|getBoolean()|, \lstinline|getInt()|, \lstinline|getString()|, usw.

Von jeder \lstinline|getXXX()|-Methode gibt es zwei Varianten. Man kann entweder
die Nummer der gewünschten Spalte als Parameter angeben (die erste Spalte hat
die Nummer 1) oder man kann als Parameter den Spaltennamen angeben. Beispiel:

\begin{lstlisting}
while (rs.next()) {                  æ// alle Datensätze der Reihe nach durchgehen 
æ  String name = rs.getString(2);     æ// der Name steht in der zweiten Spalte
æ  int tiernr = rs.getInt("tier_id"); æ// holt den Inhalt der Spalte tier_id als int 
æ}
\end{lstlisting}


\section{Datenbankänderungen}

Für Datenbankänderungen mit den SQL-Anweisungen \lstinline|INSERT INTO|,
\lstinline|UPDATE| oder \lstinline|DELETE FROM| besitzt das
\myClass{Statement}-Objekt die Methode \lstinline|executeUpdate()|:

\begin{lstlisting}
public int executeUpdate(String sql) throws SQLException
\end{lstlisting}

\lstinline|executeUpdate()| erhält genau wie \lstinline|executeQuery()| die
SQL-Anweisung als Parameter vom Typ \myClass{String}. Im Erfolgsfall wird die
Anzahl der geänderten Datensätze zurückgegeben. Beispiel:

\begin{lstlisting}
int anzahl = stmt.executeUpdate("DELETE FROM tier WHERE name='Mausi'");
\end{lstlisting}


\section{Beispiel}

\begin{lstlisting}[numbers=left, xleftmargin=7mm]
æimport java.awt.*;æ
import java.sql.*;æ
import javax.swing.*;
import javax.swing.border.EmptyBorder;

public class Tierdatenbank extends JFrame {
  private static final int WIDTH = 250;
  private static final int HEIGHT = 200;
  private DefaultListModel<String> tierliste = new DefaultListModel<String>();

  public Tierdatenbank(final String title) {
    super(title);
    createGUI();æ
    datenbankAbfrage();æ
  }
æ  
  public void datenbankAbfrage() {
    try (
      Connection conn = DriverManager.getConnection(
              "jdbc:mysql://localhost/haustier?serverTimezone=UTC&useSSL=false",
              "root", "root"); 
      Statement stmt = conn.createStatement()
    ) {
      ResultSet rs = stmt.executeQuery("SELECT name, tierart FROM tier");
      while (rs.next()) {
        tierliste.addElement(rs.getString("name") + ", " + rs.getString("tierart")); 
      }
    } catch (SQLException e) {
      JOptionPane.showMessageDialog(this, e.getMessage(),
              "Fehlermeldung", JOptionPane.ERROR_MESSAGE);
      e.printStackTrace();
    }
  }æ
  
  private void createGUI() {
    setDefaultCloseOperation(JFrame.EXIT_ON_CLOSE);
    JPanel contentPane = new JPanel();
    contentPane.setBorder(new EmptyBorder(5, 5, 5, 5));
    contentPane.setLayout(new BorderLayout(0, 0));
    contentPane.setPreferredSize(new Dimension(WIDTH, HEIGHT));
    setContentPane(contentPane);
    JLabel lblTierListe = new JLabel();
    lblTierListe.setText("Liste aller Tiere:");
    contentPane.add("North", lblTierListe);
    JList<String> listTiere = new JList<String>(tierliste);
    contentPane.add("Center", listTiere);
    pack();
    setLocationRelativeTo(null);
    setResizable(true);
    setVisible(true);  
  }
  
  public static void main(final String[] args) {
    EventQueue.invokeLater(new Runnable() {
      public void run() {
        try {
          new Tierdatenbank("Haustier-Datenbank");
        } catch (Exception e) {
          e.printStackTrace();
        }
      }
    });
  }
}æ
\end{lstlisting}