\clearpage

\lehead[]{\normalfont\sffamily\hspace*{-2.00cm}\textcolor{white}{\colorbox{lightblue}{\makebox[1.60cm][r]{\thechapter}}}\hspace{0.17cm}\textcolor{lightblue}{\chaptertitle}}
\rohead[]{\textcolor{lightblue}{\chaptertitle}\normalfont\sffamily\hspace*{0.17cm}\textcolor{white}{\colorbox{lightblue}{\makebox[1.60cm][l]{\thechapter}}}\hspace{-2.00cm}}
%\chead[]{}
\rehead[]{\textcolor{lightblue}{AvHG, Inf, My}}
\lohead[]{\textcolor{lightblue}{AvHG, Inf, My}}

\lstset{style=myJava}

\section[Allgemeines zur Implementierung von Event-Listenern
(Ereignisbehandlung)]{Allgemeines zur Implementierung von Event-Listenern\\
(Ereignisbehandlung)}

In den Musterlösungen findet ihr zwei verschiedene Ansätze um Ereignisse zu behandeln:

\begin{compactenum}[1.] 
\item (Beispiele: \myFile{Fahne.java} und \myFile{Anrede.java}) Implementierung
eines Interfaces

Dies entspricht dem Vorgehen, welches auch auf im Abschnitt
\glqq Einfache Dialogelemente\grqq\ beschrieben ist: Ein zur jeweiligen
Komponente passender EventListener wird implementiert. Dass sieht man dann schon im Kopf der 
Klasse

\begin{lstlisting}
public class ... implements ... 
\end{lstlisting}

Wie immer bei der Verwendung eines Interfaces müssen dann alle Methoden des 
Interfaces implementiert werden (in den Beispielen hier ist dies praktischer
Weise jeweils nur eine einzige Methode).

Damit die Komponente dann etwas von diesem EventListener hat, muss er auf die 
jeweilige Komponente noch \glqq angesetzt\grqq\ werden:

Beispiel aus \myFile{Anrede.java}:  

\begin{lstlisting}
btnAuswerten.addActionListener(this);
\end{lstlisting}

Beispiel aus \myFile{Fahne.java}:

\begin{lstlisting}
cbWindVonLinks.addItemListener(this);
\end{lstlisting}

  
\item (Beispiel: \myFile{FarbenMischen.java}): Ableiten von einer passenden
Wrapper-Klasse bzw. einem Interface in einer anonymen Klasse:
   
\begin{lstlisting}
button.addActionListener(new ActionListener() {
    @Override
    public void actionPerformed(final ActionEvent evt) {
        æ// Was auch immer getan werden soll ...
æ    }
});
\end{lstlisting}
   		
   
Auf den ersten Blick etwas verwirrend, aber sehr praktisch:

In der unter 1. beschriebenen Methode wurde der EventListener auf die
Klasse der Anwendung (\lstinline|this|) registriert. Wenn die
Anwendungsklasse aber das entsprechende Interface nicht implementiert, kann der
EventListener auch nicht auf diese Klasse registriert werden. Vielmehr muss
er auf eine andere Klasse registriert werden, die solch einen EventListener zur
Verfügung stellt. Da es diese Klasse noch nicht gibt, muss sie noch erschaffen
werden. Die sehr kompakte Schreibweise mit der anonymen Klasse oben ist eine
Abkürzung für

\begin{lstlisting}
button.addActionListener(MyActionListener());
   		
æ...
æ   		
class MyActionListener implements ActionListener {
    @Override
    public void actionPerformed(ActionEvent evt) {
        æ// Was auch immer getan werden soll ...
    æ}
});
\end{lstlisting}

In der kompakten Schreibweise mit der anonymen Klasse spart man es also überhaupt 
eine neue Klasse zu benennen (im gegebenen Beispiel:
\myClass{MyActionListener}). Wenn diese Klasse nur ein einziges Mal benutzt
werden soll (also nicht mehrere Objekte dieser Klasse erzeugt werden müssen),
dann bietet die Verwendung von anonymen Klassen die kompakteste Schreibweise.

Diese Art der Implementierung eines EventListeners für eine Komponente erhält
man auch, wenn den WindowBuilder in Eclipse benutzt und dort per Rechtsklick
auf eine Komponente $\rightarrow$ \myPMI{Add event handler} $\rightarrow$
\myPMI{action} $\rightarrow$ \myPMI{actionPerformed} wählt.
\end{compactenum}


\section{Typkonvertierung}

In Java besitzen alle Variablen einen Datentyp. Neben den primitiven (im
Sprachkern fest implementierten) Datentypen \lstinline|int|, \lstinline|short|,
\lstinline|long|, \lstinline|float|, \lstinline|double|, \lstinline|boolean|,
\lstinline|char| und \lstinline|byte| gibt es weitere Datentypen, die in Klassen
definiert sind bzw.\ werden.

Java erlaubt dabei die Umwandlung (\emph{type casting}) von einem Datentyp in
einen anderen. Allerdings nicht beliebig. Dabei wird unterschieden zwischen
impliziten und expliziten Typumwandlungen.

\subsection{Beispiel}

In einer Ereignis-Behandlungsroutine wird immer ein \myClass{Event}-Objekt
übergeben, das Informationen über die Art des aufgetretenen Ereignisses
besitzt. Mit Hilfe der Methode

\begin{lstlisting}
public Object getSource()
\end{lstlisting}

kann man erfragen, in welcher Komponente das Ereignis aufgetreten ist. Als
Rückgabewert erhält man ein Objekt, das keinen speziellen Typen besitzt. Wenn
man weiß, dass das Ereignis zum Beispiel in einem Button aufgetreten sein muss,
kann man das System zwingen, das allgemeine Objekt vom Typ \myClass{Object} in
ein \myClass{JButton}-Objekt umzuwandeln. Die Umwandlung geschieht jedoch „auf
eigene Gefahr“. Falls das Ereignis doch in einer anderen Komponente ausgelöst
wurde, z.B. in einem \myClass{JTextField}, gibt es einen Programmabsturz.

Der folgende Code demonstriert, wie man in einem \myClass{JFrame} mit drei
Buttons herausfinden kann, welcher der Buttons gedrückt wurde:

\begin{lstlisting}
æimport java.awt.*;
import java.awt.event.*;
import javax.swing.*;
import javax.swing.border.EmptyBorder;

public class WelcherButton extends JFrame æimplements ActionListeneræ {
    private static final int WIDTH = 300;
    private static final int HEIGHT = 120;
    private JButton btnEins, btnZwei, btnDrei;
    private JLabel lblAusgabe1, lblAusgabe2;

    public WelcherButton(final String title) {
        super(title);
        setDefaultCloseOperation(JFrame.EXIT_ON_CLOSE);
        JPanel contentPane = new JPanel();
        contentPane.setLayout(new FlowLayout());
        setContentPane(contentPane);
        setSize(WIDTH, HEIGHT);
        btnEins = new JButton("1");
        contentPane.add(btnEins);
æ        btnEins.addActionListener(this);
æ        btnZwei = new JButton("2");
        contentPane.add(btnZwei);
æ        btnZwei.addActionListener(this);
æ        btnDrei = new JButton("3");
        contentPane.add(btnDrei);
æ        btnDrei.addActionListener(this);
æ        lblAusgabe1 = new JLabel("Es wurde Button _ gedrückt.");
        contentPane.add(lblAusgabe1);
        lblAusgabe2 = new JLabel("Es wurde Button _ gedrückt.");
        contentPane.add(lblAusgabe2);
        setVisible(true);
    }

æ    public void actionPerformed(ActionEvent e) {
        JButton button = (JButton) e.getSource();

        // 1. Alternative: lange Variante
        if (button.equals(btnEins)) {
            lblAusgabe1.setText("Es wurde Button 1 gedrückt.");
        }
        if (button.equals(btnZwei)) {
            lblAusgabe1.setText("Es wurde Button 2 gedrückt.");
        }
        if (button.equals(btnDrei)) {
            lblAusgabe1.setText("Es wurde Button 3 gedrückt.");
        }

        // 2. Alternative: kurze Variante
        lblAusgabe2.setText("Es wurde Button " + button.getText() + " gedrückt.");
    }
æ
    public static void main(final String[] args) {
        new WelcherButton("WelcherButton");
    }
}
\end{lstlisting}


\subsection{Regeln für die Typkonvertierungen}

\subsubsection{Schema}

Eine Variable \lstinline|var1| vom Typ \myClass{Typ1} wird mit folgendem Befehl
explizit in eine Variable \lstinline|var2| vom Typ \myClass{Typ2} umgewandelt
(explizite Typumwandlung bzw.\ \emph{explicit type cast}):

\begin{lstlisting}
Typ2 var2 = (Typ2) var1;
\end{lstlisting}

In einigen wenigen fällen erledigt Java diese Typumwandlung für uns sogar
automatisch (implizite Typumwandlung bzw.\ \emph{implicit type cast}). Dann kann
auf den \emph{type cast operator} (im Beispiel oben \lstinline|(Typ2)| )
verzichtet werden.

In Berechnungen, in denen Ganzzahl- und Fließkommazahl-Variablen gemischt
verwendet werden geschieht dies häufig (und oft mit \glqq überraschenden\grqq\
Ergebnissen).

\subsubsection{Beispiel für die implizite Typumwandlung}

Schon bei der Umwandlung von \glqq ähnlichen\grqq\ Datentypen
(etwa \lstinline|int| $\leftrightarrow$ \lstinline|long|, oder auch
\lstinline|float| $\leftrightarrow$ \lstinline|double|) kann es Probleme geben:
Die Umwandlung des \glqq kleineren\grqq\ in den \glqq größeren\grqq\ Datentyp
ist problemlos und kann deshalb implizit erfolgen:

\begin{lstlisting}
int  i = 12345;
long l = i;
\end{lstlisting}

Für den umgekehrten Weg ist aber ein expliziter Type Cast notwendig, da hier ein
Informationsverlust stattfinden kann (der Wertebereich des \glqq größeren\grqq\
Datentyps lässt sich nicht komplett auf den \glqq kleineren\grqq\ Datentyp
abbilden). Das kann nicht automatisch passieren, sondern muss durch den
Programmierer explizit angewiesen (und damit verantwortet) werden:

\begin{lstlisting}
long l = 12345678901L;
int  i = (int) l;      æ// ergibt -539222987, weil die oberen Bytes abgeschnitten werden
\end{lstlisting}

\pagebreak

Aber auch die Umwandlung von nicht-ähnlichen Datentypen geschieht teils
implizit:

\begin{lstlisting}
int i = 5;
int j = 4;
float x = 0.2f;
System.out.println("" + x * i);              æ// ergibt 1.0
æSystem.out.println("" + i / j);              æ// ergibt 1
\end{lstlisting}

Das Ergebnis der ersten Berechnung überrascht nicht. Und trotzdem war hierfür
eine implizite Typumwandlung nötig, um die Datentypen der beiden Operanden
aneinander anzupassen. Das Ganzzahl-Objekt wurde dabei in ein Fließkomma-Objekt
umgewandelt.

Für die zweite Berechnung hätten wir intuitiv eine Typumwandlung erwartet. Aber
sie findet nicht statt. Da beide Operanden Ganzzahl-Objekte sind, sieht Java
keine Notwendigkeit für eine Umwandlung. Und das Ergebnis einer
Ganzzahl-Division ist dann eben wieder eine Ganzzahl und nicht etwa eine
Fließkommazahl (der Nachkommateil des von uns erwarteten Ergebnisses wird dabei
einfach ignoriert).

Um bei der zweiten Berechnung das von uns erwartete Ergebnis zu bekommen, müssen
wir per explizitem Type Cast zumindest einen der beiden Operanden in eine
Fließkommazahl konvertieren:

\begin{lstlisting}
System.out.println("" + ((float) i) / j);    æ// ergibt 1.25
\end{lstlisting}

\subsubsection{Regeln}

\begin{compactitem}
\item Konvertierungen zwischen \lstinline|int| und \lstinline|boolean| sind
nicht möglich. Weder implizit noch explizit.

\item Konvertierung zwischen \lstinline|int| und \lstinline|char| sind möglich:

\begin{lstlisting}
c = (char) i;     æ// explicit type cast ist nötig
æi = c;            æ// explicit type cast ist nicht nötig
\end{lstlisting}

Dies folgt der zuvor beschriebenen Logik, dass eine Umwandlung vom \glqq
größeren\grqq\ zum \glqq kleineren\grqq Datentyp einen expliziten Type Cast
benötigt (möglicher Informationsverlust!), der umgekehrte Weg aber implizit
möglich ist (keine Gefahr von Informationsverlust).

\item Konvertierung zwischen \lstinline|int| und \lstinline|String| sind
mit Type Cast nicht möglich. Weder implizit noch explizit.

\item Konvertierung zwischen \lstinline|char| und \lstinline|String| sind
mit Type Cast nicht möglich. Weder implizit noch explizit.

\item Konvertierung zwischen \lstinline|int| und einer Klasse
(zum Beispiel \myClass{JButton}) sind mit Type Cast nicht möglich. Weder
implizit noch explizit.

\item Eine Konvertierung zwischen zwei verschiedenen Klassen (zum Beispiel
\myClass{Component} und \myClass{JTextField}) ist erlaubt, wenn die eine Klasse
eine Oberklasse der anderen Klasse ist. Falls die Objekte nicht zueinander
passen, gibt es während des Programmablaufs einen Absturz.
\end{compactitem}

\subsubsection{Sonderrolle von \myClass{String}}

Die Klasse \myClass{String} nimmt in Java eine Sonderrolle ein: Einerseits
gehört sie nicht zu den primitiven Datentypen sondern ist als Klasse
implementiert. Andererseits sind ihre Eigenschaften im Java-Kern bekannt und
erlauben so einige Sachen, die mit anderen Objekten \glqq normaler\grqq\ Klassen
nicht möglich wäre. So ist es indirekt doch möglich, Zahlenwerte in Strings
umzuwandeln:

\begin{lstlisting}
int i = 10;
String text = "";
text = (String) i;                     æ// Direkte Typumwandlung nicht(!) möglich 
ætext = "Du hast " + i + " Finger.";    æ// ergibt: "Du hast 10 Finger."
ætext = "";
text += i;                             æ// ergibt: den String "10"
ætext = "Der Schreiner hat nur noch " + (i - 1) + " Finger.";
\end{lstlisting}
