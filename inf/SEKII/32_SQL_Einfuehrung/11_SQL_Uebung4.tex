\clearpage

\lehead[]{\normalfont\sffamily\hspace*{-2.00cm}\textcolor{white}{\colorbox{lightblue}{\parbox[c][0.70cm][b]{1.60cm}{
\makebox[1.60cm][r]{\thechapter}\\ \makebox[1.60cm][r]{ÜBUNG}}}}\hspace{0.17cm}\textcolor{lightblue}{\chaptertitle}}
\rohead[]{\textcolor{lightblue}{\chaptertitle}\normalfont\sffamily\hspace*{0.17cm}\textcolor{white}{\colorbox{lightblue}{\parbox[c][0.70cm][b]{1.60cm}{\thechapter\\
ÜBUNG}}}\hspace{-2.00cm}}
%\chead[]{}
\rehead[]{\textcolor{lightblue}{AvHG, Inf, My}}
\lohead[]{\textcolor{lightblue}{AvHG, Inf, My}}

\section[\hspace{1mm} SQL -- Übung 4: Die Haustierdatenbank (Fortsetzung)]{SQL -- Übung 4: Die Haustierdatenbank (Fortsetzung)}

\subsection{Aufgabe 1: Datenbank erweitern}

Sandra und Mirco Sandelmann haben sich nach einer langen Trennungsphase wieder
versöhnt. Sandra zieht wieder zu Mirco in das gemeinsame Haus zurück. Aber wem
gehören nun die Tiere? Beiden natürlich. Erweitere die Datenbank entsprechend!
Und korrigiere die Adresse von Sandra.

Die Kinder von Sandra Anderson haben sich übrigens beschwert, dass nur ihre
Mutter als Besitzerin in der Datenbank eingetragen ist. Sie möchten ebenfalls
verzeichnet werden:

Anka und Max Anderson wohnen beide bei der Mutter. Kai Anderson wohnt in der
Unsinnstraße 65, 28245 Bremen.


\subsection{Aufgabe 2: Daten abfragen}

Führe auf der erweiterten Datenbank die folgenden Abfragen durch:

\begin{compactenum}[a)]
\item Zeige eine Liste aller Besitzer (Vor- und Nachname) und ihrer Tiere (Name
und Tierart) an. Sortiere die Liste absteigend nach dem Namen der Tiere.

\item Liste alle Besitzer von Maja auf (Vor- und Nachname).

\item Zähle die Anzahl der Besitzer von Blacky.

\item Erstelle eine Liste, die angibt, wie viele Tiere jede einzelne Person
besitzt (Ausgabe: Anzahl, Vor- und Nachname).

\item Liste die vollständigen Daten aller Besitzer auf, deren Telefonnummer
nicht bekannt ist.

\item Liste Straße, PLZ und Ort aller Besitzer auf. Dabei soll jede Adresse nur
 einmal ausgegeben werden.
\end{compactenum}

\subsection{Aufgabe 3: Daten verändern}
\begin{compactenum}[a)]
\item Die Katze Mausi ist weggelaufen und hat nun offensichtlich keinen Besitzer
 mehr. Lösche die Beziehung von Mausi zu ihren beiden Besitzern Sandra und
 Mirco Sandelmann, aber lösche nicht die Katze selbst aus der Datenbank.
\item Johanna Sonntag wünscht sich sehnlich einen Hund und möchte schon vor der
 Anschaffung eines Tieres in die Datenbank eingetragen werden. Füge in die
 Besitzer-Tabelle die folgenden Daten ein:\\
 \myUserInput{Frau, Johanna, Sonntag, Glücksweg 13, 28333, Bremen}
\end{compactenum}

\subsection{Aufgabe 4: Daten abfragen 2}

Führe auf der erweiterten Datenbank die folgenden Abfragen durch:

\begin{compactenum}[a)]
\item Liste die Personen, die momentan kein Tier besitzen, mit Ihrem Vor- und
 Nachnamen auf.
\item Liste die Tiere mit Namen auf, die keinen Besitzer haben.
\item Liste alle Personen, die mit Anka Anderson zusammen wohnen (d.h. alle, die
 dieselbe Straße haben) mit ihren vollständigen Daten auf. Anka selbst darf
 auch in der Liste erscheinen.
\item Liste alle Tiere mit Namen auf, die am selben Tag geboren wurden wie Maja.
 Maja selbst soll nicht in der Liste erscheinen.
\item Liste die Namen der Tiere auf, die am selben Tag geboren wurden wie Maja,
und den Vor- und Nachnamen ihrer Besitzer. Maja selbst soll nicht in der Liste
erscheinen.
\item Liste alle Tiere mit Namen auf, die denselben Besitzer haben wie Hasso.
Hasso darf ebenfalls in der Liste erscheinen.
\end{compactenum}