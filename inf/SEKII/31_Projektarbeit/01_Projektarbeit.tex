\chapter{Projektarbeit}
\renewcommand{\chaptertitle}{Projektarbeit}

\lehead[]{\normalfont\sffamily\hspace*{-2.00cm}\textcolor{white}{\colorbox{lightblue}{\makebox[1.60cm][r]{\thechapter}}}\hspace{0.17cm}\textcolor{lightblue}{\chaptertitle}}
\rohead[]{\textcolor{lightblue}{\chaptertitle}\normalfont\sffamily\hspace*{0.17cm}\textcolor{white}{\colorbox{lightblue}{\makebox[1.60cm][l]{\thechapter}}}\hspace{-2.00cm}}
%\chead[]{}
\rehead[]{\textcolor{lightblue}{AvHG, Inf, My}}
\lohead[]{\textcolor{lightblue}{AvHG, Inf, My}}

\lstset{style=myJava}

Die Projektarbeit wird als Ersatz für eine der beiden Klausuren des Halbjahres
gewertet. Wenn ihr zu zweit oder dritt programmiert, soll jeder einen Teil des
Codes alleine schreiben (einige Code-Abschnitte schreibt man sicher am besten
zusammen). Im Quellcode muss gekennzeichnet sein, wer welchen Code programmiert
hat.

Für die Projektarbeit sollt ihr ein eigens dafür anzulegendes Repository
benutzen, auf das die am Projekt beteiligten schreibenden Zugriff und ich
Leserechte habe.

Da ihr nun nicht mehr alleine in einem Repository arbeitet, ist es \emph{sehr}
zu empfehlen, jeden Commit durch einen knappen, aber aussagekräftigen Kommentar
zu versehen (der Commit-Dialog in Eclipse bietet das an). Der Commit-Kommentar
sollte für jede geänderte/hinzugefügte Datei entsprechende Hinweise geben. Etwa
so:

\begin{lstlisting}
Spieler.java:
- Klasse neu angelegt

Spielfeld.java: 
- Wert für Breite ist nun abhängig von der gegebenen Höhe (Seitenverhältnis 4:3)
- neue Methode: resize()
\end{lstlisting}

Auch wenn euch dies zunächst lästig erscheinen mag: Im Laufe der Projektarbeit
werdet ihr so viel effizienter zusammen arbeiten können. Denn mit Hilfe dieser
Commit-Kommentare lässt sich auch im Nachhinein schnell überblicken, wer wann
was geändert hat. Bei der Fehlersuche kann dies ein unschätzbarer Vorteil sein!

\section{Programm-Planung}

Abgabetermin:       spätestens \ldots (wird festgelegt)

Abzugeben sind:

\begin{compactitem}
\item Ein Text, der beschreibt, was das Programm leisten und wie es aussehen
soll. Dies kann eventuell eine Vorversion der Programm-Anleitung sein.

\item Eine Skizze von der geplanten Programm-Oberfläche.

\item Ein Programmentwurf mit einem UML-Klassendiagramm und eventuell einem oder
mehreren Zustandsdiagrammen. Im Klassendiagramm soll gekennzeichnet sein, wer
welche Klasse programmiert.
\end{compactitem}

\textbf{Tipp}: Denkt daran, dass das Programm auch fertig werden muss! Plant
deshalb lieber erst mal eine ganz einfache Version des Programms. Falls am Ende
noch Zeit vorhanden ist, könnt ihr das Programm dann noch weiter ausbauen.


\section{Programmierung und Test}

Abgabetermin: ca.\ drei bis vier Wochen nach dem Termin für die Abgabe der
Programm-Planung (wird festgelegt)

Abzugeben sind:

\begin{compactitem}
\item Der Quellcode des Programms (im Repository). Der Quellcode muss
folgendermaßen mit Javadoc kommentiert sein:

\begin{compactitem}
\item In jeder Datei muss ein Javadoc-Kommentar stehen, der angibt, welchem
Zweck die Klasse dient.
\item Jede Methode und jede globale Variable einer Klasse müssen mit einem
Javadoc-Kommentar versehen werden, der ihren Zweck erklärt (es sei denn, der
Name ist selbsterklärend).
\item Hilfsvariablen, die innerhalb einer Methode deklariert sind, brauchen
nicht kommentiert werden.
\item Es muss gekennzeichnet sein, wer welchen Code programmiert hat.
\end{compactitem}

\item Ein UML-Klassendiagramm, das alle tatsächlich programmierten Klassen und
die Beziehungen zwischen den Klassen darstellt. Attribute und Methoden dürfen
in diesem Diagramm weggelassen werden (Das Klassendiagramm, das zur Planung
dient, soll dagegen auch die wichtigen Attribute und Methoden enthalten).

\item Eine Bedienungsanleitung (am besten in Form eines Dialogs als Teil des
Programms).

\end{compactitem}

\textbf{Wichtig}: Sobald ihr mich über die Fertigstellung per e-Mail informiert,
spätestens aber zum vereinbarten Abgabetermin, dürft ihr solange keine weiteren
Änderungen am Repository mehr vornehmen, bis ich euch entsprechend informiert
habe.


\section{Benotung}

\begin{compactenum}[a)]
\item  Programmiertechnik (50\% der Endnote)

Übersichtlichkeit, Effizienz und Schwierigkeitsgrad der Programmierung.
Angemessener Einsatz der im Unterricht besprochenen Konzepte (Klassen,
Vererbung, Arrays, Fehlerbehandlung, usw.). Selbständige Arbeitsweise.
Am wichtigsten: Das Programm muss fehlerfrei arbeiten.

\item Spiel-Gestaltung (25\% der Endnote)

Programmidee; klar verständliche Bedienoberfläche (Bedienungsfreundlichkeit);
optische Wirkung.

\item Dokumentation (25\% der Endnote)

Vollständigkeit, Korrektheit und Verständlichkeit der Dokumentation. Dies
bezieht sich sowohl auf die Planung wie auch auf die Endversion.
\end{compactenum}



