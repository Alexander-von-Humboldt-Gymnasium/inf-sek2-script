\chapter{Sound}
\renewcommand{\chaptertitle}{Sound}

\lehead[]{\normalfont\sffamily\hspace*{-2.00cm}\textcolor{white}{\colorbox{lightblue}{\makebox[1.60cm][r]{\thechapter}}}\hspace{0.17cm}\textcolor{lightblue}{\chaptertitle}}
\rohead[]{\textcolor{lightblue}{\chaptertitle}\normalfont\sffamily\hspace*{0.17cm}\textcolor{white}{\colorbox{lightblue}{\makebox[1.60cm][l]{\thechapter}}}\hspace{-2.00cm}}
%\chead[]{}
\rehead[]{\textcolor{lightblue}{AvHG, Inf, My}}
\lohead[]{\textcolor{lightblue}{AvHG, Inf, My}}

\lstset{style=myJava}

Analog zum Laden von Bildern können in Java auch Sound-Dateien geladen werden.

\section{Wiedergabe von *.wav Dateien}

\begin{lstlisting}
import javax.sound.sampled.AudioSystem;
import javax.sound.sampled.Clip;
import javax.sound.sampled.LineUnavailableException;
import javax.sound.sampled.UnsupportedAudioFileException;

Clip clip = AudioSystem.getClip();
clip.open(AudioSystem.getAudioInputStream(getClass().getResource("ok.wav")));
\end{lstlisting}

Anschließend kann man das \myClass{Clip} Objekt wahlweise mit
\lstinline|loop(Clip.LOOP_CONTINUOUSLY)| sich ständig wiederholend, oder mit 
\lstinline|start()| einmalig,  abspielen lassen.

Mit \lstinline|stop()| lässt sich die Wiedergabe stoppen. Der Programmablauf
wartet nicht auf \lstinline|start()| bzw. \lstinline|loop()| sondern wird sofort
fortgesetzt.

Komplettes Beispiel:

\begin{lstlisting}
import javax.sound.sampled.AudioSystem;
import javax.sound.sampled.Clip;
import javax.sound.sampled.LineUnavailableException;
import javax.sound.sampled.UnsupportedAudioFileException;
import java.io.IOException;æ
import java.awt.BorderLayout;
import java.awt.EventQueue;
import javax.swing.JFrame;
import javax.swing.JPanel;
import javax.swing.border.EmptyBorder;

public class PlayWAV extends JFrame {
  private JPanel contentPane;

  public PlayWAV() {
    super("PlayWAV");
    setDefaultCloseOperation(JFrame.EXIT_ON_CLOSE);
    setBounds(100, 100, 450, 300);
    contentPane = new JPanel();
    contentPane.setBorder(new EmptyBorder(5, 5, 5, 5));
    contentPane.setLayout(new BorderLayout(0, 0));
    setContentPane(contentPane);
    æClip sound;
    try {
      sound = AudioSystem.getClip();
      sound.open(AudioSystem.getAudioInputStream(getClass().getResource("ok.wav")));
      sound.start();
      //sound.loop(Clip.LOOP_CONTINUOUSLY);
    } catch (LineUnavailableException e) {
      e.printStackTrace();
    } catch (IOException e) {
      e.printStackTrace();
    } catch (UnsupportedAudioFileException e) {
      e.printStackTrace();
    }
æ  }

  public static void main(String[] args) {
    EventQueue.invokeLater(new Runnable() {
      public void run() {
        try {
          PlayWAV frame = new PlayWAV();
          frame.setVisible(true);
        } catch (Exception e) {
          e.printStackTrace();
        }
      }
    });
  }
}
\end{lstlisting}


\begin{minipage}{1.0\textwidth}
\section{Wiedergabe von *.mp3 Dateien}

\begin{lstlisting}
æimport java.awt.EventQueue;æ
import java.util.concurrent.CountDownLatch;æ
import javax.swing.JFrame;æ
import javafx.embed.swing.JFXPanel;
import javafx.scene.media.Media;
import javafx.scene.media.MediaPlayer;æ

public class PlayMP3 extends JFrame {æ
  private MediaPlayer mediaPlayer;  
  // Das MediaPlayer Objekt nuss als Attribut der Klasse deklariert 
  // werden, sonst wird die Wiedergabe des Stücks nach ein paar Sekunden
  // abgebruchen, weil der Garbage Collector das MediaPlayer
  // Objekt abräumt ... 
æ
  public PlayMP3() {
    super("PlayMP3");
    setDefaultCloseOperation(JFrame.EXIT_ON_CLOSE);
    setBounds(100, 100, 450, 300);æ
    playClip();æ
  }
æ
  public void playClip() {
    Media clip = new
         Media(getClass().getResource("Every_OS_Sucks.mp3").toString());
    mediaPlayer = new MediaPlayer(clip);
    mediaPlayer.play();
  }æ

  public static void main(String[] args) {
    æfinal CountDownLatch latch = new CountDownLatch(1);
    EventQueue.invokeLater(new Runnable() {
      public void run() {
        new JFXPanel(); // initializes JavaFX environment
        latch.countDown();
      }
    });
    try {
      latch.await();
    } catch (InterruptedException e) {
      e.printStackTrace();
    }æ
    EventQueue.invokeLater(new Runnable() {
      public void run() {
        try {
          PlayMP3 frame = new PlayMP3();
          frame.setVisible(true);
        } catch (Exception e) {
          e.printStackTrace();
        }
      }
    });
  }
}
\end{lstlisting}
\end{minipage}


Für die MP3-Wiedergabe muss auf das JavaFX-Framework
zurückgegriffen werden. Da JavaFX nicht mehr mehr Teil des openJDK (und auch nicht 
mehr Teil von Oracles JDK) ist, muss es separat eingebunden werden. Die dazu 
nötigen Schritte sind hier nicht erklärt. Erste Anlaufstelle ist die JavaFX 
Projekt-Homepage: \url{https://openjfx.io/}

So können übrigens nicht nur \myFile{*.mp3} sondern auch \myFile{*.wav} Dateien
wiedergegeben werden. Wenn man vorab also noch nicht weiß welches Dateiformat
später zum Einsatz kommt, ist man so auf der sicheren Seite.

Wenn es andererseits klar ist, dass man mit \myFile{*.wav} Dateien auskommt,
dann ist der zuerst vorgestellte Weg über \myClass{AudioClip} der deutlich
einfachere.
