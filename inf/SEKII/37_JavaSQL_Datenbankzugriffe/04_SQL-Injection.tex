\clearpage

\lehead[]{\sf\hspace*{-2.00cm}\textcolor{white}{\colorbox{lightblue}{\makebox[1.60cm][r]{\thechapter}}}\hspace{0.17cm}\textcolor{lightblue}{\chaptertitle}}
\rohead[]{\textcolor{lightblue}{\chaptertitle}\sf\hspace*{0.17cm}\textcolor{white}{\colorbox{lightblue}{\makebox[1.60cm][l]{\thechapter}}}\hspace{-2.00cm}}
%\chead[]{}
\rehead[]{\textcolor{lightblue}{AvHG, Inf, My}}
\lohead[]{\textcolor{lightblue}{AvHG, Inf, My}}

\lstset{style=myJava}


\section{SQL-Injection}

Unter \emph{SQL-Injection} versteht man das Einschleusen von SQL-Befehlen
in eine Datenbankanwendung durch einen Angreifer. Dabei handelt es sich um eine
der populärsten Angriffsmethoden (nicht nur) im Internet.

Der Angreifer nutzt dabei aus, dass bei Benutzereingaben Metazeichen wie etwa
das Hochkomma oder auch Kommentarzeichen nicht überprüft bzw.\ maskiert werden.

Auf diese Weise kann er 

\begin{compactitem}
\item direkt oder indirekt Informationen über die Struktur der
Datenbank erhalten.
\item Zugang zu einem System erlangen.
\item Daten in der Datenbank verändern.
\item Daten, Tabellen oder Datenbanken löschen.
\item Dateien im Dateisystem des Datenbankservers erzeugen oder überschreiben.
\end{compactitem}

Ein hoffentlich anschauliches Beispiel findest du im Kursrepository, im Ordner
\myFile{37\_JavaSQL\_Datenbankzugriffe} findet ihr einen weiteren Ordner:
\myFile{sqlinjection}. Wenn ihr dessen Inhalt in euer eigenes Repository
kopiert, könnt ihr mit dem Programm \myUserInput{OnlineBanking} ein wenig \glqq spielen
\grqq . Zuvor müsst ihr allerdings das SQL-Skript \myFile{online-banking.sql}
einmal ausführen.

In der kleinen Datenbank ist genau ein Datensatz vorhanden. Das Konto mit der
Kontonummer 1234567 ist mit dem Passwort 1234 versehen. Kontoinhaber ist Frau
Monika Mustermann. Der aktuelle Kontostand beträgt -703,28 \texteuro .

Mit dem kleinen Programm könnt ihr nur zwei Sachen tun: Ihr könnt euch anmelden
und ihr könnt das Passwort ändern. Wenn ihr angemeldet seid, wird euch der
aktuelle Kontostand angezeigt.

In der Datei \myFile{Angriffsmöglichkeiten.txt} öffnet, findet ihr einige
Beispiel dafür, wie ihr mit Hilfe von SQL-Injections das Programm missbrauchen
könnt!

Wenn du es dir genauer anschaust, wirst du erkennen, warum die beschriebenen
Angriffe erfolgreich sind.

Glücklicherweise, kannst du als Java-Programmierer solche Angriffe leicht
abwehren. Dazu musst du nur die Klasse \myClass{PreparedStatement}, statt der
Klasse \myClass{Statement} verwenden:

Aus

\begin{lstlisting}
Statement stmt = conn.createStatement();
String kontonummer = tfKontonummer.getText();
String passwort = tfPasswort.getText();
String sql = "SELECT * FROM konto WHERE kontonummer = " 
        + kontonummer + " AND passwort = '" + passwort + "'";
System.out.println(sql);
ResultSet rs = stmt.executeQuery(sql);
\end{lstlisting}

wird

\begin{lstlisting}
æString sql = æ"SELECT * FROM konto WHERE kontonummer = ? AND passwort = ?"æ;
æPreparedStatement ps = conn.prepareStatement(sql); 
ps.setString(1, tfKontonummer.getText());
ps.setString(2, tfPasswort.getText());
æSystem.out.println(æps.toString()æ);
ResultSet rs = æps.executeQuery()æ;
\end{lstlisting}

Dem vorbereiteten SQL-Befehl mussten vor der Ausführung nur noch die passenden
Argumente an Stelle der Fragezeichen verpasst werden.

Die Arbeit mit \lstinline|executeUpdate()| funktioniert analog zu diesem
Beispiel. Du kannst dir die entsprechend korrigierten Java-Klassen im
Unterordner \myFile{preparedstatement} anschauen. Damit laufen die SQL-Injection
Angriffe alle ins Leere.
