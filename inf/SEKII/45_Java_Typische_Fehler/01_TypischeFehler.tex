\chapter{Typische Fehler \ldots\ und wie man sie vermeidet}
\renewcommand{\chaptertitle}{Typische Fehler \ldots\ und wie man sie vermeidet}

\lehead[]{\normalfont\sffamily\hspace*{-2.00cm}\textcolor{white}{\colorbox{lightblue}{\makebox[1.60cm][r]{\thechapter}}}\hspace{0.17cm}\textcolor{lightblue}{\chaptertitle}}
\rohead[]{\textcolor{lightblue}{\chaptertitle}\normalfont\sffamily\hspace*{0.17cm}\textcolor{white}{\colorbox{lightblue}{\makebox[1.60cm][l]{\thechapter}}}\hspace{-2.00cm}}
%\chead[]{}
\rehead[]{\textcolor{lightblue}{AvHG, Inf, My}}
\lohead[]{\textcolor{lightblue}{AvHG, Inf, My}}

\lstset{style=myJava}

Programmieren ist ein Handwerk. Und da kann man wie überall Fehler machen.
Einige Fehler passieren aus Unerfahrenheit oder wegen mangelnder Übung. Dazu
kommen noch Flüchtigkeitsfehler.

Außerdem erkennen \glqq Anfänger\grqq\ üblicherweise noch nicht den Wert von
Namenskonventionen und einheitlicher Formatierung (vor Allem der Einrückung) von
Quelltexten, was im Resultat dazu führt, dass die geschriebenen Programme
unübersichtlicher und schwerer verständlich werden. Eben weil das eigene
Verständnis für die Programme beim Anfänger zwangsläufig noch nicht weit
entwickelt ist, fallen ihm die so selbst verursachten zusätzlichen Probleme
zunächst gar nicht auf. Der Aufwand, der beispielsweise mit sauberem Einrücken
und vernünftiger und Standard konformer Wahl geeigneter Namen für Klassen,
Variablen und Methoden einhergeht, erscheint so üblicherweise übertrieben.

Schließlich gibt es noch Mängel in der Struktur von Programmen.

In diesem Kapitel will ich auf die typischen Fehler eingehen. Einige davon
lassen sich leicht vermeiden. Andere zumindest leicht aufspüren. Auch dazu - zum
Aufspüren von Fehlern (\emph{Debugging}) will ich hier einige Tipps geben.


\section{Debugging: Fehler finden}

Fehler passieren. Anfängern, aber auch Fortgeschrittenen und auch Profis.
Deshalb ist es wichtig zu lernen, wie man Fehler möglichst schnell findet.

\subsection{Debugging mit einfachen Mitteln: Textausgabe auf der Konsole}

Ein geeignetes und leicht einzusetzendes Mittel ist die Ausgabe von Text auf der
sogenannten \emph{Konsole} (Falls es den \myPMI{Console}-View bei dir in Eclipse
noch nicht gibt, kannst du ihn mit \myPMI{Window} $\rightarrow$ \myPMI{Show
View} $\rightarrow$ \myPMI{Console} hinzufügen).

Wenn du nun in einem Programm die Anweisung

\begin{lstlisting}
System.out.println("Irgend ein Text");
\end{lstlisting}

schreibst, so wird im Console-View genau dieser Text ausgegeben. Das ist
nützlich, wenn dein Programm sich nicht so verhält, wie du es erwartest. Dann
kannst du über diese \lstinline|System.out.println()| Anweisungen an geeigneter
Stelle in deinem Programm dafür sorgen, dass du das Verhalten des Programms
während dessen Ausführung besser nachvollziehen kannst.

Im einfachsten Fall möchtest du einfach überprüfen, ob ein bestimmter Teil
deines Programms überhaupt durchlaufe wird. Etwa, ob eine von dir geschriebene
Methode überhaupt ausgeführt wird:

\begin{lstlisting}
æprivate int kleinsteZahl(int zahl1, int zahl2) {æ
  System.out.println("kleinsteZahl(): wurde aufgerufen");æ
  if (zahl1 < zahl2) {
    return zahl1;
  } else {
    return zahl2;
  }
}
\end{lstlisting}

Oft ist es aber auch hilfreich bei dieser Gelegenheit auch gleich noch den
Inhalt einer oder auch mehrerer Variablen zu überprüfen:

\begin{lstlisting}
æprivate int kleinsteZahl(int zahl1, int zahl2) {æ
  System.out.println("kleinsteZahl(): zahl1 = " + zahl1 + ",   zahl2 = " + zahl2);æ
  if (zahl1 < zahl2) {
    return zahl1;
  } else {
    return zahl2;
  }
}
\end{lstlisting}



\section{Fehlende geschweifte Klammern}

Mehrere Anweisungen können durch geschweifte Klammern zu einem Anweisungsblock
zusammengefasst werden. So werden Programme strukturiert: Welche Anweisungen
gehören zur Methodendefinition? Welche Anweisungen sollen durch die
Kontrollstruktur gesteuert werden?

Beispiel für einen typischen Anfängerfehler:

\begin{lstlisting}
int i = 0;
int j = 10;
if (i > 0)
  i = i + 1;
  j = i;
System.out.println("j hat den Wert " + j);
\end{lstlisting}

Welche Ausgabe ist zu erwarten? Die Einrückung des Quelltexts in diesem Beispiel
lässt vermuten, dass zumindest der Autor dieses Programms folgende Ausgabe
erwartet:

\begin{lstlisting}
j hat den Wert 10
\end{lstlisting}

Tatsächlich wird die Ausgabe aber lauten

\begin{lstlisting}
j hat den Wert 0
\end{lstlisting}

Grund sind die fehlenden geschweiften Klammern. So hat es der Autor des
fehlerhaften Beispiels oben vermutlich gemeint:

\begin{lstlisting}
int i = 0;
int j = 10;
if (i > 0) {
  i = i + 1;
  j = i;
}
System.out.println("j hat den Wert " + j);
\end{lstlisting}

Erst durch die geschweiften Klammern werden die beiden eingerückten Zeilen zu
einem Anweisungsblock, der als ganzer in Abhängigkeit vom \lstinline|if|
ausgeführt wird -- oder eben nicht. Ohne die geschweiften Klammern wirkt das
\lstinline|if| nur auf die nächste Anweisung (\lstinline|i = i + 1;|). Die
zweite Anweisung (\lstinline|j = i;|) wird unabhängig vom \lstinline|if| auf jeden Fall
ausgeführt!


\section{Das tödliche Semikolon}

Gerade in der E-Phase sehe ich diesen Fehler relativ oft: Eine Kontrollstruktur
(\lstinline|if|, \lstinline|while| oder \lstinline|for|) wird direkt durch ein
Semikolon \glqq kurzgeschlossen\grqq . Beispiel:

\begin{lstlisting}
int i = 0;
while (i < 10); {
  i = i + 1;
}
\end{lstlisting}

Welchen Wert wird \lstinline|i| anschließend haben? Würde ich diese Frage in der
E-Phase stellen, würden mir viele Schüler antworten \glqq Neun\grqq. Andere
vielleicht mit \glqq Zehn\grqq .

Beides ist falsch! Tatsächlich gibt es gar kein \emph{anschließend}, denn die
\lstinline|while|-Schleife wird endlos ausgeführt.

Schuld ist das Semikolon direkt hinter dem Schleifenkopf: Alle
Kontrollstrukturen funktionieren so, dass die nächste Anweisung bzw.\
der folgende Anweisungsblock gesteuert wird.

Das Semikolon direkt hinter einer Kontrollstruktur ist aber eine \emph{leere
Anweisung}. Im Beispiel oben kontrolliert der Schleifenkopf auch nur diese
eine leere Anweisung, nicht aber -- wie sicher intendiert -- den folgenden
Anweisungsblock.

\subsection{\ldots und wie man es vermeiden kann}

Der Eclipse Editor kann den Quelltext nach (konfigurierbaren) Regeln
formatieren. Dazu benutzt man entweder die Tastenkombination
\myUserInput{<Strg>-<Umschalten>-F} oder ruft diese Funktion über das Menü auf:
\myPMI{Source} $\rightarrow$ \myPMI{Format}.

Dann hätte man schon am resultierenden Quellcode sehen können, wo das Problem
liegt:

\begin{lstlisting}
int i = 0;
while (i < 10)
  ; 
{
  i = i + 1;
}
\end{lstlisting}

Auch die fehlenden geschweiftem Klammern aus dem vorangegangenen Abschnitt wären
so sofort augenfällig geworden:

\begin{lstlisting}
int i = 0;
int j = 10;
if (i > 0)
  i = i + 1;
j = i;
System.out.println("j hat den Wert " + j);
\end{lstlisting}