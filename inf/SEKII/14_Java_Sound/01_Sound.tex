\chapter{Sound}
\renewcommand{\chaptertitle}{Sound}

\lehead[]{\normalfont\sffamily\hspace*{-2.00cm}\textcolor{white}{\colorbox{lightblue}{\makebox[1.60cm][r]{\thechapter}}}\hspace{0.17cm}\textcolor{lightblue}{\chaptertitle}}
\rohead[]{\textcolor{lightblue}{\chaptertitle}\normalfont\sffamily\hspace*{0.17cm}\textcolor{white}{\colorbox{lightblue}{\makebox[1.60cm][l]{\thechapter}}}\hspace{-2.00cm}}
%\chead[]{}
\rehead[]{\textcolor{lightblue}{AvHG, Inf, My}}
\lohead[]{\textcolor{lightblue}{AvHG, Inf, My}}

\lstset{style=myJava}

Analog zum Laden von Bildern können in Java auch Sound-Dateien geladen werden.

\section{Wiedergabe von *.wav Dateien}

\begin{lstlisting}
import java.applet.Applet;
import java.applet.AudioClip;
AudioClip sound = Applet.newAudioClip(getClass().getResource("ok.wav"));
\end{lstlisting}

Anschließend kann man das \myClass{AudioClip} Objekt wahlweise mit
\lstinline|play()| einmalig, oder mit \lstinline|loop()| sich ständig
wiederholend abspielen lassen. 

Mit \lstinline|stop()| lässt sich die Wiedergabe stoppen. Der Programmablauf
wartet nicht auf \lstinline|play()| bzw. \lstinline|loop()| sondern wird sofort
fortgesetzt.

Komplettes Beispiel:

\begin{lstlisting}
import java.applet.Applet;
import java.applet.AudioClip;æ
import java.awt.BorderLayout;
import java.awt.EventQueue;
import javax.swing.JFrame;
import javax.swing.JPanel;
import javax.swing.border.EmptyBorder;

public class PlayWAV extends JFrame {
  private JPanel contentPane;

  public PlayWAV() {
    super("PlayWAV");
    setDefaultCloseOperation(JFrame.EXIT_ON_CLOSE);
    setBounds(100, 100, 450, 300);
    contentPane = new JPanel();
    contentPane.setBorder(new EmptyBorder(5, 5, 5, 5));
    contentPane.setLayout(new BorderLayout(0, 0));
    setContentPane(contentPane);
    æAudioClip sound = Applet.newAudioClip(getClass().getResource("ok.wav"));
    sound.play();
    //sound.loop();
æ  }

  public static void main(String[] args) {
    EventQueue.invokeLater(new Runnable() {
      public void run() {
        try {
          PlayWAV frame = new PlayWAV();
          frame.setVisible(true);
        } catch (Exception e) {
          e.printStackTrace();
        }
      }
    });
  }
}
\end{lstlisting}


\begin{minipage}{1.0\textwidth}
\section{Wiedergabe von *.mp3 Dateien}

\begin{lstlisting}
æimport java.awt.BorderLayout;
import java.awt.EventQueue;æ
import java.util.concurrent.CountDownLatch;æ
import javax.swing.JFrame;
import javax.swing.JPanel;
import javax.swing.border.EmptyBorder;æ
import javafx.embed.swing.JFXPanel;
import javafx.scene.media.Media;
import javafx.scene.media.MediaPlayer;æ

public class PlayMP3 extends JFrame {
  private JPanel contentPane;

  public PlayMP3() {
    super("PlayMP3");
    setDefaultCloseOperation(JFrame.EXIT_ON_CLOSE);
    setBounds(100, 100, 450, 300);
    contentPane = new JPanel();
    contentPane.setBorder(new EmptyBorder(5, 5, 5, 5));
    contentPane.setLayout(new BorderLayout(0, 0));
    setContentPane(contentPane);
    playClip();
  }

  public void playClip() {
    æMedia clip = new
         Media(getClass().getResource("Every_OS_Sucks.mp3").toString());
    MediaPlayer mediaPlayer = new MediaPlayer(clip);
    mediaPlayer.play();æ
  }

  public static void main(String[] args) {
    æfinal CountDownLatch latch = new CountDownLatch(1);
    EventQueue.invokeLater(new Runnable() {
      public void run() {
        new JFXPanel(); // initializes JavaFX environment
        latch.countDown();
      }
    });
    try {
      latch.await();
    } catch (InterruptedException e) {
      e.printStackTrace();
    }æ
    EventQueue.invokeLater(new Runnable() {
      public void run() {
        try {
          PlayMP3 frame = new PlayMP3();
          frame.setVisible(true);
        } catch (Exception e) {
          e.printStackTrace();
        }
      }
    });
  }
}
\end{lstlisting}
\end{minipage}


Für die MP3-Wiedergabe muss auf das noch recht neue JavaFX-Framework
zurückgegriffen werden. Die nötige Laufzeitumgebung wird erst seit JDK 7 Update
6 standardmäßig mit installiert. Die Bibliothek \myFile{jfxrt.jar} muss in
Eclipse noch in das Projekt eingebunden werden. Seit JDK 8 ist das nicht mehr
nötig.

So können übrigens nicht nur \myFile{*.mp3} sondern auch \myFile{*.wav} Dateien
wiedergegeben werden. Wenn man vorab also noch nicht weiß welches Dateiformat
später zum Einsatz kommt, ist man so auf der sicheren Seite.

Wenn es andererseits klar ist, dass man mit \myFile{*.wav} Dateien auskommt,
dann ist der zuerst vorgestellte Weg über \myClass{AudioClip} der deutlich
einfachere.
