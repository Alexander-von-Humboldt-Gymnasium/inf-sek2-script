%%----------------------------------------------------------------------
% (c) 2008 by
%   Raoul Rubien - rubienr-at-sbox-dot-tugraz-dot-at
%   Thomas Quaritsch - tq-qt-sbox-dot-tugraz-dot-at
%
% Template Version 0.36 (April 2008)
%%----------------------------------------------------------------------
\newcommand{\templateVersion}    {0.36/Apr. 2008}     % template version

%%----------------------------------------------------------------------
%
% ToDo: Adjust this details
% Switch to non draft mode when your document is ready. Note that images
% are not shown while \documentclass - draft is on, wheras overfull
% boxes are.
%
%\documentclass[a4paper, 11pt, fleqn, parskip, draft]{scrartcl}
\documentclass[BCOR=20mm,DIV=20,twoside,parskip=half,headinclude,ngerman,fontsize=10pt]{scrreprt}
\newcommand{\isDraft}          {false} % true | false
\usepackage[utf8]{inputenc}
\usepackage[T1]{fontenc}
\usepackage{babel} 
%\defaulthyphenchar=127
\renewcommand{\familydefault}{\sfdefault}
%\usepackage[headsepline]{scrpage2}  % Lade Linie zw Kopf und Text
\usepackage{scrpage2}
\pagestyle{empty}
\cfoot[]{\pagemark}   % Seitenzahl zentriert im Seitenfuss, bei plain 
\rofoot[]{}
\lefoot[]{}		   % keine Seitenzahl aussen im Seitenfuss
%\lehead{\Large\textbf{Eclipse als Java-Entwicklungsumgebung}}
%\rehead{AvHG, Inf, My}
%\lohead{\Large\textbf{Eclipse als Java-Entwicklungsumgebung}}
%\rohead{AvHG, Inf, My}
\setkomafont{pageheadfoot}{\normalfont\sffamily}  % Serifenlose Schrift
\usepackage{helvet}
\usepackage[lighttt]{lmodern}
\KOMAoptions{DIV=last}
%\usepackage{xcolor}
\usepackage{ifthen}                   % controllflow
\usepackage{xcolor}
%\definecolor{lightblue}{cmyk}{0.346, 0.114, 0, 0.106}
%%----------------------------------------------------------------------
%
% useful color definitions
%
\definecolor{white}              {rgb}{1,1,1}                % R:255 G:255 B:255
\definecolor{paleblue}           {rgb}{0.211,0.373,0.603}    % R: 76 G:129 B:178
\definecolor{lightblue}          {rgb}{0.3,0.5,0.7}          % R: 54 G: 96 B:154
\definecolor{mauve}{rgb}         {0.58,0,0.82}

\usepackage{graphicx}
\usepackage{setspace}
\usepackage{microtype}
%\usepackage{amsmath}
\usepackage{textcomp}

\usepackage{framed} 
\colorlet{shadecolor}{gray!25} 

\usepackage{paralist}        % compactenum und compactitem:
\setlength{\pltopsep}{2mm}   % Abstände vor erstem item
\setlength{\plitemsep}{1mm}  % zwischen items und
\setlength{\plparsep}{1mm}   % zwischen absätzen innerlb eines items

\usepackage{array}
\usepackage{footnote}  % footnotes in tabular

\usepackage{tabularx}

\usepackage{caption}

%\usepackage[table]{pstricks}
%\usepackage{pst-node, pst-tree, pst-dbicons, pst-eps}

%\usepackage[user,titleref]{zref} 
 

\usepackage[
    pdftex=true,             %% sets up hyperref for use with the pdftex program
    plainpages=false,        %% set it to false, if pdflatex complains: 
                             %%``destination with same identifier already exists''
    backref,                 %% adds a backlink text to the end of each biblio item
    pagebackref=false,       %% if true, creates backward references as a list of 
                             %% page numbers in the bibliography
    colorlinks=true,         %% turn on colored links (true is better for on-screen
                             %% reading, false is better for printout versions)
    bookmarks=true,          %% if true, generate PDF bookmarks (requires two passes 
                             %% of pdflatex)
    bookmarksopen=false,     %% if true, show all PDF bookmarks expanded
    bookmarksnumbered=false, %% if true, add the section numbers to the bookmarks
    %pdfstartpage={1},       %% determines, on which page the PDF file is opened
    %pdfpagelayout=TwoPageRight,
    %pdfpagemode=FullScreen,        %% None, UseOutlines (=show bookmarks),
    % UseThumbs
                             %% (show thumbnails), FullScreen
    linkcolor=black, 
    urlcolor=black,
    citecolor=black,
    anchorcolor=black,
    pagecolor=black,
    unicode=true,
    pdftitle={AvHG -- Informatik Oberstufe},
    pdfauthor={Hartmut Meyer <h.meyer6@schule.bremen.de>}
  ]{hyperref}
  
%%----------------------------------------------------------------------
%
% section styles
%
\makeatletter
\renewcommand{\chapter}{                          % the style
  \@startsection {chapter}                        % the name
  {1}                                             % the level
  {0mm}                                           % the indent
  {0mm}                                           % the before skip
  {0.1\baselineskip}                              % the after skip
  {\rm\LARGE\bfseries\textcolor{paleblue}}        % fontstyle
}

\renewcommand{\section}{                          % the style
  \@startsection {section}                        % the name
  {2}                                             % the level
  {0mm}                                           % the indent
  {-1.6\baselineskip}                             % the before skip
  {0.1\baselineskip}                              % the after skip
  {\rm\Large\bfseries\textcolor{paleblue}}        % fontstyle
}

\renewcommand{\subsection}{                       % subsection style
  \@startsection {subsection}                     % the name
  {3}                                             % the level
  {0mm}                                           % the indent
  {-1.6\baselineskip}                             % the before skip
  {0.1\baselineskip}                              % the after skip
  {\rmfamily\bfseries\Large\textcolor{lightblue}} % fontstyle
}

\renewcommand{\subsubsection}{                    % subsection style
  \@startsection {subsubsection}                  % the name
  {4}                                             % the level
  {0mm}                                           % the indent
  {-0.8\baselineskip}                             % the before skip
  {0.1\baselineskip}                              % the after skip
  {\normalfont\normalsize\sffamily\textbf}        % fontstyle
}
\makeatother


%%----------------------------------------------------------------------
% some helper
%%----------------------------------------------------------------------
% insert highlighted TODOs
% \toDo {text}
\newcommand{\toDo}[1]
{
  \textcolor{red}{#1}
}

\usepackage{listings}

% Program Menu Item
\newcommand{\myPMI}[1]{\textit{#1}}

% File Name
\newcommand{\myFile}[1]{\texttt{#1}}

% Class Name
\newcommand{\myClass}[1]{\texttt{\textbf{#1}}}

% Package Name
\newcommand{\myPackage}[1]{\texttt{#1}}

% Kommando Name
\newcommand{\myCmd}[1]{\texttt{#1}}

% User Input
\newcommand{\myUserInput}[1]{\texttt{#1}}

\newcommand{\leadingzero}[1]{\ifnum #1<10 0\the#1\else\the#1\fi}
\newcommand{\todayI}{\the\year-\leadingzero{\month}-\leadingzero{\day}}


\lstdefinestyle{mySQL}{
	backgroundcolor=\color{gray!10},
	basewidth={0.5em},
	basicstyle=\ttfamily\small,
	commentstyle=\ttfamily\small,
	emph={IF,SCHEMA,SHOW,USE,AUTO_INCREMENT,DOUBLE,ENUM,DATETIME,TABLES,IS,XOR,REFERENCES},
	emphstyle={\bfseries},
	escapeinside=\`\`,
	language=SQL,
	literate=
		{Ö}{{\"O}}1
		{Ä}{{\"A}}1
		{Ü}{{\"U}}1
		{ß}{{\ss}}2
		{ü}{{\"u}}1
		{ä}{{\"a}}1
		{ö}{{\"o}}1,
	moredelim=**[is][\color{gray}]{æ}{æ},
	numberstyle=\footnotesize,
	showspaces=false,
	showstringspaces=false,
	%stringstyle=\color{mauve},
	tabsize=2,
}

\lstdefinestyle{myJava}{
	backgroundcolor=\color{gray!10},
	basewidth={0.5em},
	basicstyle=\ttfamily\bfseries\normalsize,
	commentstyle=\ttfamily\normalsize,
	emph={},
	emphstyle={\bfseries},
	escapeinside=\`\`,
	language=Java,
	literate=
		{Ö}{{\"O}}1
		{Ä}{{\"A}}1
		{Ü}{{\"U}}1
		{ß}{{\ss}}2
		{ü}{{\"u}}1
		{ä}{{\"a}}1
		{ö}{{\"o}}1,
	moredelim=**[is][\color{gray}]{æ}{æ},
	numberstyle=\footnotesize,
	showspaces=false,
	showstringspaces=false,
	%stringstyle=\color{gray},
	tabsize=2,
}

%%----------------------------------------------------------------------
%
% author details
% ToDo: Adjust this details
%
\newcommand{\Author}             {Vera Lüthje, Hartmut Meyer}          % author
% details \newcommand{\AuthorMatnum}       {0123456}                   % author's matricula number
\newcommand{\AuthorEmail}       {h.meyer6@schule.bremen.de}       
% author's email address
\newcommand{\course} {AvHG, Inf, My} %
\newcommand{\method}             {Scientific Writer}         % mode is shown within top header
\newcommand{\lang}               {ger}                       % language (ger|eng)
\newcommand{\toc}                {true}                     % table of contents
% before \tof (true | false)
\newcommand{\tof}                {false}                     % table of figures at the very last page (true | false)

