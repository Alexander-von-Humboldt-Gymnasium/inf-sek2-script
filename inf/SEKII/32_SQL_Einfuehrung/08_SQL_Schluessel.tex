\clearpage

\lehead[]{\sf\hspace*{-2.00cm}\textcolor{white}{\colorbox{lightblue}{\makebox[1.60cm][r]{\thechapter}}}\hspace{0.17cm}\textcolor{lightblue}{\chaptertitle}}
\rohead[]{\textcolor{lightblue}{\chaptertitle}\sf\hspace*{0.17cm}\textcolor{white}{\colorbox{lightblue}{\makebox[1.60cm][l]{\thechapter}}}\hspace{-2.00cm}}
%\chead[]{}
\rehead[]{\textcolor{lightblue}{AvHG, Inf, My}}
\lohead[]{\textcolor{lightblue}{AvHG, Inf, My}}

\section{Tabellen verknüpfen: Primär- und Fremdschlüssel}

\subsection{Primärschlüssel}

Um Datensätze einer Tabelle eindeutig identifizieren zu können, braucht man
einen sogenannten \emph{Schlüssel}. Als Schlüssel kann jedes einzelne Attribut
oder -- wenn nötig -- auch eine Kombination mehrerer Attribute der Tabelle
dienen. Wichtig ist nur, dass sich der Schlüssel \emph{konzeptionell} zur
eindeutigen Identifizierung eines jeden Datensatzes der Tabelle eignet.
So reicht es etwa nicht, zu sehen, dass in unserer Haustierdatenbank keine
Kombination aus Tiername und Tierart doppelt vorkommt. Denn niemand kann
garantieren, dass in Zukunft nicht doch noch ein weiteres Tier mit einer zuvor
bereits existierenden Kombination aus Tierart und Tiername hinzukommen wird.
Tiername und Tierart sind also zusammen kein geeigneter Schlüssel für die
Tabelle \lstinline|tier|!

An einen Schlüssel wird zudem die Anforderung gestellt, dass er \emph{minimal}
sein soll: alle Attribute, die man zum Schlüssel dazu zählt müssen auch wirklich
\emph{nötig} sein, um die eindeutige Identifizierung aller Datensätze (auch
konzeptionell) sicher zu stellen.

Bei einer gegebenen Tabelle können durchaus verschiedene Attribute oder
Kombinationen aus Attributen als Schlüssel in Frage kommen. Den Schlüssel, den
man aus der Menge der möglichen Schlüsselkandidaten zur weiteren Verwendung
auswählt, bezeichnet man als \emph{Primärschlüssel} (engl.\ \emph{Primary Key}).

Als Primärschlüssel sind besonders gut extra für diesen Zweck erzeugte Attribute
vom Typ \lstinline|INT| geeignet, die von der Datenbank mit Hilfe des
Schlüsselwortes \lstinline|AUTO_INCREMENT| beim Erzeugen eines neuen Datensatzes
automatisch mit einem bisher noch nicht vergebenen Wert belegt werden.

Überhaupt ist es empfehlenswert, als Primärschlüssel ein Attribut zu benutzen,
welches keine sonstigen Information trägt! Damit umgeht man so manche Falle!
Beispielsweise könnte man annehmen, das die ISBN-Nummer ein geeigneter
Primärschlüssel für eine \lstinline|buch|-Tabelle wäre. Oder das die
Postleitzahl ein ebenso geeigneter Primärschlüssel für eine
\lstinline|ort|-Tabelle wäre. Beides stellt sich bei näherer Betrachtung als
falsch heraus (es gibt Postleitzahlen, aus denen nicht eindeutig auf einen
bestimmten Ort geschlossen werden kann und es gibt auch ISBN-Nummern auf die
nicht eindeutig auf eine bestimmte Ausgabe eines Buches geschlossen werden
kann). Durch die Verwendung eines nicht-informationstragenden Primärschlüssels,
umgeht man solche Fallen. Zudem muss man sich keine Sorgen machen, dass es in
Zukunft aus inhaltlichen Gründen notwendig werden könnte die Werte des
Primärschlüsselattributes zu ändern -- so wie es beispielsweise bei den
Postleitzahlen mehrfach (zuletzt 1993) geschehen ist.

\begin{lstlisting}
æCREATE TABLE besitzer
(æ
	`\textbf{\texttt{besitzer\_id INT AUTO\_INCREMENT,}}`
æ	anrede ENUM('Herr', 'Frau', 'Firma') DEFAULT 'Herr',
	vorname VARCHAR(20),
	nachname VARCHAR(20) NOT NULL,æ
	`\textbf{\texttt{PRIMARY KEY (besitzer\_id)}}`
æ);
\end{lstlisting}


\subsection{Fremdschlüssel}

Als \emph{Fremdschlüssel} (engl.\ \emph{Foreign Key}) bezeichnet man ein
Attribut einer Tabelle, welches auf die Werte eines Primärschlüssels einer
anderen Tabelle verweist. Genau genommen muss der Primärschlüssel noch nicht
einmal aus einer anderen Tabelle stammen: auch Verweise auf den Primärschlüssel
der eigenen Tabelle sind möglich!

\begin{lstlisting}
æCREATE TABLE tier
(
	name VARCHAR(20) NOT NULL,
	tierart VARCHAR(20) NOT NULL,
	lebendig ENUM('ja','nein') DEFAULT 'ja',
	geschlecht ENUM('weiblich','männlich') DEFAULT 'weiblich',
	geburtstag DATE,
	todestag DATE,æ
	`\textbf{\texttt{tier\_besitzer\_id INT,}}`
	`\textbf{\texttt{FOREIGN KEY (tier\_besitzer\_id) REFERENCES besitzer (besitzer\_id)}}` 
æ);
\end{lstlisting}

Hinter \lstinline|REFERENCES| wird dabei die Tabelle und das entsprechende
Attribut benannt, auf welches sich der Fremdschlüssel bezieht. Dadurch ist es
auch sehr leicht, in einem gegebenen SQL-Skript die entsprechenden Beziehungen
zwischen den Tabellen zu erkennen.


\subsection{Referentielle Integrität}

Indem man der Datenbank mitteilt, dass es sich bei einem Attribut um einen
Fremdschlüssel handelt, versetzt man sie auch in die Lage über die sogenannte
\emph{referentielle Integrität} der Datenbank zu wachen. Die Datenbank
verhindert dann, dass ein neuer Datensatz so angelegt wird, dass der Wert für
den Fremdschlüssel nicht durch einen bereits existierenden Wert im
Primärschlüsselfeld der referenzierten Tabelle gedeckt ist. Im Beispiel unserer
Haustierdatenbank: Nachdem wir das Attribut \lstinline|tier_besitzer_id| in der
\lstinline|tier|-Tabelle als Fremdschlüssel definiert haben, der auf den
Primärschlüssel der \lstinline|besitzer|-Tabelle verweist, wird es die Datenbank
nicht erlauben, dass wir ein neues Tier anlegen, mit einem Wert für
\lstinline|tier_besitzer_id|, den es so noch nicht als Wert für den
Primärschlüssel der \lstinline|besitzer|-Tabelle gibt. Ebenso würde verhindert,
dass ein Datensatz aus der \lstinline|besitzer|-Tabelle gelöscht oder so
verändert würde, dass in der Folge ein Datensatz in der \lstinline|tier|-Tabelle
mit einem Verweis auf einen nicht länger existierenden Wert im Primärschlüssel
der \lstinline|besitzer|-Tabelle resultieren würde.



\subsection{Verknüpfung von Tabellen}

Nicht immer, aber sehr oft, werden bei Datenbankabfragen (\lstinline|SELECT|)
Daten aus mehreren Tabellen verknüpft. Diese Verknüpfung erfolgt über die
Verbindung von Primär- und Fremdschlüsseln.

In der Ergebnisliste der Abfrage interessieren uns nur solche Ergebnisse, die
sich aus zusammengehörigen Datensätzen der beteiligten Tabellen ergeben. Wenn
beispielsweise eine Ausgabe aller Tiere und ihrer Besitzer verlangt ist, dann
sollen natürlich für jedes Tier auch wirklich nur die passenden Besitzer
ausgegeben werden und nicht irgendwelche Datensätze aus der
\lstinline|besitzer|-Tabelle. Und genau dies erreichen wir über eine geeignete
Bedingung, die den Fremdschlüssel der einen Tabelle mit dem passenden
Primärschlüssel der anderen Tabelle verknüpft:

\begin{lstlisting}
æSELECT tier.name, tier.tierart, besitzer.vorname, besitzer.nachname
FROM tier, besitzeræ
`\textbf{\texttt{WHERE tier.tier\_besitzer\_id = besitzer.besitzer\_id}}`;
\end{lstlisting}

Bisher kennen wir nur den Fall mit maximal zwei beteiligten Tabellen. Sobald in
einer Abfrage noch weitere Tabellen hinzukommen, ist es nötig, die
entsprechenden Primärschlüssel-Fremdschlüssel-Beziehungen aller beteiligter
Tabellen in der \lstinline|WHERE|-Klausel zu benennen. Das kann bei mehr als
drei beteiligten Tabellen durchaus unübersichtlich werden. Wenn ihr in der
Ergebnisliste zu viele (und dabei auch unsinnige Kombinationen) erhaltet, ist
das ein sicherer Hinweis darauf, dass ihr nicht alle Beziehungen in der
\lstinline|WHERE|-Klausel berücksichtigt habt!
