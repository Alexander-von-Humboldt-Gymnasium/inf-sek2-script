\clearpage

\lehead[]{\sf\hspace*{-2.00cm}\textcolor{white}{\colorbox{lightblue}{\makebox[1.60cm][r]{\thechapter}}}\hspace{0.17cm}\textcolor{lightblue}{\chaptertitle}}
\rohead[]{\textcolor{lightblue}{\chaptertitle}\sf\hspace*{0.17cm}\textcolor{white}{\colorbox{lightblue}{\makebox[1.60cm][l]{\thechapter}}}\hspace{-2.00cm}}
%\chead[]{}
\rehead[]{\textcolor{lightblue}{AvHG, Inf, My}}
\lohead[]{\textcolor{lightblue}{AvHG, Inf, My}}

\section[\hspace{1mm} SQL -- Daten abfragen (Fortsetzung)]{SQL -- Daten abfragen (Fortsetzung)}

\subsection{Kartesisches Produkt}

Die Ergebnistabelle, die alle möglichen Zeilen aus der Kombination zweier
Tabellen enthält, wird als das \textit{Kartesische Produkt} aus zwei Tabellen
bezeichnet.

Das Kartesische Produkt von einer Tabelle \myUserInput{a} mit einer Tabelle
\myUserInput{b} unter Auswahl sämtlicher Spalten erhält man mit der
SQL-Anweisung:

\begin{lstlisting}
SELECT * FROM a, b;
\end{lstlisting}

Das Ergebnis ist natürlich unsinnig. Statt dessen brauchen wir die Verknüpfung
über die Fremdschlüssel-Primärschlüssel-Beziehung zwischen den beteiligten
Tabellen:

\begin{lstlisting}
SELECT * FROM a, b
WHERE fremdschlüssel_der_einen_tabelle = primärschlüssel_der_anderen_tabelle;
\end{lstlisting}


\subsection{LEFT JOIN und RIGHT JOIN}

Gegeben sind die beiden Tabellen:

%\begin{table}[h]
%\hfill
\begin{minipage}{0.30\textwidth}
\myUserInput{besitzer}:\\ \\
\begin{tabular}{|r|l|}
\hline
\textbf{\texttt{besitzer\_id}} & \textbf{\texttt{name}} \\ \hline
\texttt{1} & \texttt{Anna} \\ \hline
\texttt{2} & \texttt{Andrew} \\ \hline
\texttt{3} & \texttt{Annabel} \\ \hline
\end{tabular}
\end{minipage}\hfill
\begin{minipage}{0.55\textwidth}
\myUserInput{fahrzeug}:\\ \\
\begin{tabular}{|l|r|}
\hline
\textbf{\texttt{fahrzeug}} & \textbf{\texttt{fahrzeug\_besitzer\_id}} \\ \hline
\texttt{Fahrrad} & \texttt{2} \\ \hline
\texttt{Auto} & \texttt{4} \\ \hline
\texttt{Auto} & \texttt{3} \\ \hline
\end{tabular}
\end{minipage}
%\hfill
%\end{table}

Die folgenden drei Abfragen demonstrieren die Wirkung eines
\lstinline{LEFT JOIN} oder eines \lstinline{RIGHT JOIN} im Vergleich zu einer
normalen Verknüpfung:

\begin{lstlisting}
SELECT * FROM besitzer, fahrzeug
WHERE besitzer_id = fahrzeug_besitzer_id;
\end{lstlisting}

ergibt:

\begin{tabular}{|r|l|l|r|}
\hline
\textbf{\texttt{besitzer\_id}} & \textbf{\texttt{name}} &
\textbf{\texttt{fahrzeug}} & \textbf{\texttt{fahrzeug\_besitzer\_id}}\\
\hline 
\texttt{2} & \texttt{Andrew} & \texttt{Fahrrad} & \texttt{2}\\
\hline
\texttt{3} & \texttt{Annabel} & \texttt{Auto} & \texttt{3}\\
\hline
\end{tabular}

\begin{lstlisting}
SELECT * FROM besitzer LEFT JOIN fahrzeug
ON besitzer_id = fahrzeug_besitzer_id;
\end{lstlisting}

ergibt:

\begin{tabular}{|r|l|l|r|}
\hline
\textbf{\texttt{besitzer\_id}} & \textbf{\texttt{name}} &
\textbf{\texttt{fahrzeug}} & \textbf{\texttt{fahrzeug\_besitzer\_id}}\\
\hline
\texttt{1} & \texttt{Anna} & \verb|[NULL]| & \verb|[NULL]|\\
\hline 
\texttt{2} & \texttt{Andrew} & \texttt{Fahrrad} & \texttt{2}\\
\hline
\texttt{3} & \texttt{Annabel} & \texttt{Auto} & \texttt{3}\\
\hline
\end{tabular}

\begin{lstlisting}
SELECT * FROM besitzer RIGHT JOIN fahrzeug 
ON besitzer_id = fahrzeug_besitzer_id;
\end{lstlisting}

ergibt:

\begin{tabular}{|r|l|l|r|}
\hline
\textbf{\texttt{besitzer\_id}} & \textbf{\texttt{name}} &
\textbf{\texttt{fahrzeug}} & \textbf{\texttt{fahrzeug\_besitzer\_id}}\\
\hline 
\texttt{2} & \texttt{Andrew} & \texttt{Fahrrad} & \texttt{2}\\
\hline
\verb|[NULL]| & \verb|[NULL]| & \texttt{Auto} & \texttt{4}\\
\hline
\texttt{3} & \texttt{Annabel} & \texttt{Auto} & \texttt{3}\\
\hline
\end{tabular}


\subsection{Verknüpfung einer Tabelle mit sich selbst}

Gegeben ist die Tabelle \myUserInput{angestellter}:

\begin{tabular}{|l|l|}
\hline
\textbf{\texttt{name}} & \textbf{\texttt{job}} \\ \hline
\texttt{Jan Bauer} & \texttt{Buchhalter} \\ \hline
\texttt{Talina Wendel} & \texttt{Programmierer} \\ \hline
\texttt{Noah Richter} & \texttt{Programmierer} \\ \hline
\texttt{Andrea Faust} & \texttt{Redakteur} \\ \hline
\end{tabular}

Um herauszufinden, wer denselben Job hat wie Talina Wendel, kann man die folgende Abfrage stellen:

\begin{lstlisting}
SELECT kollege.name
FROM angestellter talina, angestellter kollege
WHERE talina.name = 'Talina Wendel' AND kollege.job = talina.job;
\end{lstlisting}

Es wird zweimal die Tabelle \myUserInput{angestellter} verwendet. Um die beiden
Tabellen unterscheiden zu können erhalten sie Namen (ein sogenanntes Alias).
\myUserInput{talina} bezeichnet die erste Angestellten-Tabelle und
\myUserInput{kollege} die zweite Angestellten-Tabelle. Die Alias Bezeichner
dürfen frei gewählt werden und es empfiehlt sich hier ganz besonders Namen zu
wählen, die erkennen lassen, was gemeint ist.

\textbf{Aufgabe:} Welche Ausgabe erzeugt die Abfrage?


\subsection{Unterabfragen}

Ein sehr nützliches SQL-Konstrukt sind die sogenannten \emph{Unterabfragen}
(auch \emph{Sub-Selects} oder \emph{Sub-Queries} genannt). Dabei wird
innerhalb eines SQL-Befehls eine weitere SQL-Abfrage benutzt.

Beispiele dafür findest du in den Musterlösungen für die Aufgaben Übung 3,
Aufgabe 1 und Übung 4 Aufgabe 1. Etwa:

\begin{lstlisting}
UPDATE tier
SET tier_besitzer_id = (
    SELECT besitzer_id FROM besitzer
    WHERE vorname = 'Mirco'
    AND nachname = 'Sandelmann'
)
WHERE name = 'Harald';
\end{lstlisting}
