\clearpage

\lehead[]{\normalfont\sffamily\hspace*{-2.00cm}\textcolor{white}{\colorbox{lightblue}{\parbox[c][0.70cm][b]{1.60cm}{
\makebox[1.60cm][r]{\thechapter}\\ \makebox[1.60cm][r]{ÜBUNG}}}}\hspace{0.17cm}\textcolor{lightblue}{\chaptertitle}}
\rohead[]{\textcolor{lightblue}{\chaptertitle}\normalfont\sffamily\hspace*{0.17cm}\textcolor{white}{\colorbox{lightblue}{\parbox[c][0.70cm][b]{1.60cm}{\thechapter\\
ÜBUNG}}}\hspace{-2.00cm}}
%\chead[]{}
\rehead[]{\textcolor{lightblue}{AvHG, Inf, My}}
\lohead[]{\textcolor{lightblue}{AvHG, Inf, My}}

\section{Normalisierung -- Übung}

\subsection{Aufgabe 1}

Die folgende Tabelle beschreibt in welcher Abteilung die Mitarbeiter einer Firma
arbeiten und an welchem Projekt sie wie viele Stunden gearbeitet haben:

\begin{tabular}{|r|r|r|r|r|r|r|r|}\hline
\textbf{personal\_id} & \textbf{name} & \textbf{vorname} & \textbf{abt\_id} &
\textbf{abteilung} & \textbf{proj\_id} & \textbf{proj-beschreibung} &
\textbf{zeit}
\\ \hline
1 & Lorenz & Sophia & 1 & Personal & 2 & Verkaufspromotion & 83
\\ \hline 
2 & Hohl & Tatjana & 2 & Einkauf & 3 & Konkurrenzanalyse & 29
\\ \hline
3 & Willschrein & Theodor & 1 & Personal & 1, 2, 3 & Kundenumfrage, & 140,
\\
 & & & & & & Verkaufspromotion, & 92,
\\
 & & & & & & Konkurenzanalyse & 110
\\ \hline
4 & Richter & Hans-Otto & 3 & Verkauf & 2 & Verkaufspromotion & 67
\\ \hline
5 & Wiesenland & Brunhilde & 2 & Einkauf & 1 & Kundenumfrage & 160
\\ \hline
\end{tabular}

Wandle das Datenbankschema nacheinander in die erste, zweite und dritte
Normalform um.


\subsection{Aufgabe2}

Gegeben ist die Tabelle \myUserInput{bestellung} mit den folgenden Spalten:

\myUserInput{kunde\_id}, \myUserInput{kunde\_name},
\myUserInput{kunde\_adresse}, \myUserInput{bestellung\_id},
\myUserInput{bestellung\_datum}, \myUserInput{artikel\_id},
\myUserInput{artikel\_name}, \myUserInput{artikel\_anzahl}.

In einer Bestellung können mehrere Artikel gleichzeitig bestellt werden. Die
Artikel-Anzahl gibt an, wie viele Artikel einer Sorte in einer Bestellung
angefordert werden.

Überführe das Datenbankschema in die dritte Normalform.